\section{17. Übungsblatt}

\paragraph{Aufgabe 17.1}

\begin{enumerate}

\item[]

Es gilt $\alpha+\beta=\gamma$, dann

$\tan\alpha=\frac{x}{h_2}$, $\tan\gamma=\frac{x}{h_2-h_1}$

$\Rightarrow$ $\gamma-\beta=\alpha(x)=\arctan\frac{x}{h_2-h_1}-\arctan\frac{x}{h_2}$

\begin{align*}
\alpha'(x)&=\frac{1}{1+(\frac{x}{h_2-h_1})^2}\cdot\frac{1}{h_2-h_1}-\frac{1}{1+(\frac{x}{h_2})^2}\cdot\frac{1}{h_2}\\
&=\frac{(h_2-h_1)^2}{h_2-h_1}\frac{1}{(h_2-h_1)^2+x^2}-\frac{h_2^2}{h_2}\frac{1}{h_2^2+x^2}\\
&=(h_2-h_1)\cdot\frac{1}{(h_2-h_1)^2+x^2}-h_2\frac{1}{h_2^2+x^2}\underset{lok.min}{=}0
\end{align*}

$\Leftrightarrow$ $x=\pm\sqrt{h_2^2-h_1h_2}$ nehmen wir $+$, $\approx134,7cm$\\

$\alpha''(x)
\left\{
\begin{array}{rcl}
>0 & lok.Min\\
=0 & ??\\
<0 & lok.Max 
\end{array}
\right.$

\end{enumerate}

\paragraph{Aufgabe 17.2}

\begin{enumerate}

\item[]

$f(x)\geq f'(x)\geq 0$ $\Rightarrow$ $f$ ist mon. ansteigend

$f(x_0)=0$

$\forall x<x_0:\ 0\leq f(x)\leq f(x_0)=0$ $\Rightarrow$ $f(x)=0$\\

$g(x)=e^{-x}f(x)$, weil $e^{-x}>0,f(x)\geq0\Rightarrow g(x)\geq0$

\begin{equation*}
g'(x)=e^{-x}f'(x)-g(x)=e^{-x}f'(x)-e^{-x}f(x)=\underset{>0}{e^{-x}}\underset{\leq0}{(f'(x)-f(x))}
\end{equation*}

$\Rightarrow$ $g'(x)<0$, $g(x)$ mon. fall.

\end{enumerate}

\newpage

\paragraph{Aufgabe 17.4}

\begin{enumerate}

\item[]

$f(x)=x^3+ax^2+bx$

Sei $a,b\in\mathbb{R}$, ist $\overline{x}$ ein lok. Extremastelle, so gilt $f'(\overline{x})=0$

wegen $f'(x)=3x^2+2ax+b$ folgt

\begin{align*}
f'(x)=0&\Leftrightarrow x^2+\frac{2ax}{3}+\frac{b}{3}=0\\
&\Leftrightarrow x=-\frac{a}{3}\pm\sqrt{\frac{a^2}{9}-\frac{b}{3}}=-\frac{a}{3}\pm\frac{1}{3}\sqrt{a^2-3b}
\end{align*}

$f''(x)=6x+2a=2\sqrt{a^2-3b}$

\end{enumerate}

\newpage

\paragraph{Aufgabe 17.5(H)}

\begin{enumerate}

\item[(a)]

$f_1(x)=\sin(\cos(x))$

Es ist für $x\in\mathbb{R}$ differenzierbar.

\begin{equation*}
f'_1(x)=-\sin(x)\cdot\cos(\cos(x))
\end{equation*}

\item[(b)]

$f_2(x)=x\cdot|x|$

Es ist auch für $x\in\mathbb{R}$ differenzierbar.

\begin{equation*}
f'_2(x)=
\left\{
\begin{array}{rcl}
2x & when & x\geq 0\\
-2x & when & x\leq 0
\end{array}
\right.
\end{equation*}

\item[(c)]

$f_3(x)=\arctan(x)+\arctan(\frac{1}{x})$

Es ist für $x\in(\mathbb{R}\setminus{0})$ differenzierbar.

\begin{equation*}
f'_2(x)=\frac{1}{1+x^2}-\frac{1}{x^2}\frac{1}{1+(\frac{1}{x})^2}=0,\mbox{ when }x\neq0
\end{equation*}

\item[(d)]

$f_4(x)=x^{(x^x)}$

Es ist für $x\in\mathbb{R}_{\geq0}$ differenzierbar.

\begin{equation*}
f'_4(x)=\mbox{nein, wir müssen zuerst umformen}
\end{equation*}
\begin{align*}
x^{(x^x)}&=(e^{\ln(x)})^{(x^x)}\\
&=(e^{\ln(x)})^{(e^{x\ln(x)})}\\
&=e^{(e^{x\ln(x)})\ln(x)}
\end{align*}

Dann:
\begin{equation*}
f'_4(x)=(x^x(\ln(x)+1)\cdot\ln(x)+x^{x-1})\cdot e^{(e^{x\ln(x)})\ln(x)}
\end{equation*}

\end{enumerate}

\newpage

\paragraph{Aufgabe 17.6(H)}

\begin{enumerate}

\item[(a)]

\begin{proof}
$ $\newline

Sei $f(x)$ gerade, sei $g(x)=-x$, sei $h(x)=f(g(x))$

Es gilt
\begin{equation*}
h(x)=f(g(x))=f(-x)=f(x)
\end{equation*}
Dann
\begin{align*}
h'(x)&=g'(x)f'(g(x))\\
&=-f'(-x)=f'(x)\\
&\Rightarrow\ f'\ \mbox{ungerade}
\end{align*}
\end{proof}

\begin{proof}
$ $\newline

Sei $f(x)$ ungerade, $g(x)=-x$, $h(x)=f(g(x))$

Es gilt
\begin{equation*}
h(x)=f(g(x))=f(-x)=-f(x)
\end{equation*}
Dann
\begin{align*}
h'(x)&=g'(x)f'(g(x))\\
&=-f'(-x)=-f'(x)\\
&\Rightarrow\ f'\ \mbox{gerade}
\end{align*}
\end{proof}

\newpage

\item[(b)]

\begin{enumerate}

\item[(i)]

Wir beweisen nun, dass
\begin{equation*}
f:\ x\mapsto\sum_{i=0}^n a_ix^i
\end{equation*}
für alle $a_k=0$ mit $k=ungerade$ gerade ist.

\begin{proof}
$ $\newline

$\mathbf{IA}$:
sei $n=0$, es gilt
\begin{equation*}
f_0:\ x\mapsto a_0\ (gerade)
\end{equation*}

$\mathbf{IV}$:
\begin{equation*}
f:\ x\mapsto\sum_{i=0}^n a_ix^i
\end{equation*}
für alle $a_k=0$ mit $k=ungerade$ gerade ist.

$\mathbf{IS}$:
sei $\tilde{n}=n+2$, dann
\begin{align*}
&\tilde{f}:\ x\mapsto\sum_{i=0}^{\tilde{n}} a_ix^i\\
&\tilde{f}:\ x\mapsto\sum_{i=0}^{n+2} a_ix^i\\
&\tilde{f}:\ x\mapsto(\sum_{i=0}^{n} a_ix^i)+a_{n+2}x^{n+2}\\
\Rightarrow & \tilde{f}=f+a_{n+2}x^{n+2}
\end{align*}
Die Voraussetzung ist, dass mit $n=ungerade$ alle Terme $a_{n_{ungerade}}=0$ sind, dann entfernen wir alle $a_nx^n$ von ungerade $n$.

Dann ist $\tilde{f}=f+a_{n+2}x^{n+2}$ gerade, denn alle $n+2$ sind auch gerade.
\end{proof}

\newpage

\item[(ii)]

Wir beweisen nun, dass
\begin{equation*}
f:\ x\mapsto\sum_{i=0}^n a_ix^i
\end{equation*}
für alle $a_k=0$ mit $k=gerade$ ungerade ist.

\begin{proof}
$ $\newline

$\mathbf{IA}$:
sei $n=1$, es gilt
\begin{equation*}
f_0:\ x\mapsto a_1x^1\ (ungerade)
\end{equation*}

$\mathbf{IV}$:
\begin{equation*}
f:\ x\mapsto\sum_{i=0}^n a_ix^i
\end{equation*}
für alle $a_k=0$ mit $k=gerade$ ungerade ist.

$\mathbf{IS}$:
sei $\tilde{n}=n+2$, dann
\begin{align*}
&\tilde{f}:\ x\mapsto\sum_{i=0}^{\tilde{n}} a_ix^i\\
&\tilde{f}:\ x\mapsto\sum_{i=0}^{n+2} a_ix^i\\
&\tilde{f}:\ x\mapsto(\sum_{i=0}^{n} a_ix^i)+a_{n+2}x^{n+2}\\
\Rightarrow & \tilde{f}=f+a_{n+2}x^{n+2}
\end{align*}
Die Voraussetzung ist, dass mit $n=gerade$ alle Terme $a_{n_{gerade}}=0$ sind, dann entfernen wir alle $a_nx^n$ von gerade $n$.

Dann ist $\tilde{f}=f+a_{n+2}x^{n+2}$ ungerade, denn alle $n+2$ sind auch ungerade.
\end{proof}

\end{enumerate}

\end{enumerate}

\newpage

\paragraph{Tutorium}

\begin{enumerate}

\item[]

Lokale Extrema:

$D_f\subseteq \mathbb{R}$ offen, $f:D_f\rightarrow\mathbb{R}$\\

offene Menge:

$B(x,\epsilon)$: $(x-\epsilon,x+\epsilon)$

$B(x,\epsilon)\subseteq(a,b)$\\

$x_0\in d_f$ heißt lokale Min/Max von $f$, falls gilt:

$\exists\epsilon>0$, $\forall y\in B(x_0,\epsilon)$:

$f(x_0):\leq f(y)/\geq f(y)$\\

$D_f\subseteq \mathbb{R}$ offen, $f:D_f\rightarrow\mathbb{R}$ diffbar.

$x_0\in D_f$ sei lok. Extremasstelle von $f$

$\Rightarrow$ $f'(x_0)=0$

\end{enumerate}
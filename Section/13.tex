\section{13. Übungsblatt}

\subsection{13.1}

Zeigen, dass $\log_a$ und $\exp_a$ umgekehrt sind

wir haben

\begin{equation*}
\log_a x=\frac{\log x}{\log a}
\end{equation*}

\begin{align*}
\forall x\in\mathbb{R}:\log_a(\exp_a(x))&=\log_a(a^x)\\
&=\frac{\log(\exp(x\log a))}{\log a}\\
&=\frac{x\log a}{\log a}\\
&=x
\end{align*}
umgekehrt nach $y$ auch gültig

\subsection{13.2}

\subsection{13.3}

\begin{equation*}
\lim_{x\rightarrow0}\frac{e^x-1}{x}=1,\lim_{x\rightarrow0}(1+x)^{\frac{1}{x}}=e
\end{equation*}

\paragraph{(a)}

\begin{equation*}
\lim_{x\downarrow0}x^x=\lim_{x\downarrow0}\exp(\log(x^x))=\exp(\lim_{x\downarrow0}x\log(x))=e^0=1
\end{equation*}

\paragraph{(b)}

Umschreiben $\sqrt[n]{n}=n^\frac{1}{n}$

\paragraph{(c)}

Umschreiben $\frac{\log(1+x)}{x}=\log((1+x)^\frac{1}{x})$, dann offenbar $\lim=1$

\paragraph{(d)}

Umschreiben
\begin{equation*}
\frac{3x+5-2(x+3)}{(x+3)(3x+5)}=\frac{x-1}{(x+3)(3x+5)}
\end{equation*}
dann
\begin{equation*}
\frac{1}{(x+3)(3x+5)}
\end{equation*}

\paragraph{(e)}

\begin{equation*}
1-\sqrt{f(x)}=\frac{1-f(x)}{1+\sqrt{f(x)}}
\end{equation*}

$\frac{1}{2}$

\paragraph{(f)}

\begin{align*}
x\sqrt{1+\frac{1}{x^2}}&\overset{x>0}{=}\sqrt{x^2+1}\rightarrow1\\
&\overset{<0}{=}-\sqrt{x^2+1}...\rightarrow -1
\end{align*}

\newpage

\subsection{13.4}

$f(xy)=f(x)f(y)$\\

Frage:

$f(0)=?$

$f(n\in\mathbb{N})=?$

$f(z\in\mathbb{Z})=?$

$f(x\in\mathbb{Q})=?$

$f(x\in\mathbb{R})=?$\\

$1.$

$f(0)=f(0)+f(0)$ $\Rightarrow$ $f(0)=0$\\

$2.$

$f(n\in\mathbb{N})=1\cdot n$, $f(x)=ax\Rightarrow a=f(1)$\\

dann

$f(0)=0$

$f(n\in\mathbb{N})=nf(1)$ $\Leftarrow$ Induktion

$f(z\in\mathbb{Z})=zf(1)$ $\Leftarrow$ $\oplus$ $f(0)=f(n)+f(-n)$ $\Rightarrow$ $f(-n)=-f(n)=-nf(1)$

$f(x\in\mathbb{Q})=xf(1)$ $\Leftarrow$ $\exists m\in\mathbb{Z},n\in\mathbb{N}$ sodass $x=\frac{m}{n}$, dann
\begin{align*}
z.z.\ f(\frac{m}{n}&=\frac{m}{n}f(1))\\
f(m)=nf(\frac{m}{n})=nf(\frac{m}{n})&=mf(1)=f(m)
\end{align*}

$f(x\in\mathbb{R})=xf(1)$ $\Leftarrow$ (tut) $\exists(x_n)\in\mathbb{Q}$ sodass $\lim_{n\rightarrow\infty}x_n=x\in\mathbb{R}$\\

$Folgerung:$\\

1. $f(0)=0$ $\Rightarrow$ $f(-x)=-f(x)$ $\forall x\in\mathbb{R}$\\

2. $\forall x\in\mathbb{R},n\in\mathbb{N}$, $f(nx)=nf(x)$

$n=1$, $f(1x)=1f(x)=f(x)$ gültig

... $\Rightarrow$ Induktion\\

3. $\forall x\in\mathbb{R},z\in\mathbb{Z}$:

i) $z\in\mathbb{N}$ (nach 2. klar)

ii) $z<0$: $-z\in\mathbb{N}$ $\Rightarrow$
\begin{align*}
f(z)&=f((-1)(-z))=(-z)f(-1)\\
(1.)&=(-z)(-f(1))=zf(1)
\end{align*}

4. $\forall x\in\mathbb{Q}$ $\exists m\in\mathbb{Z},n\in\mathbb{N}$, $x=\frac{m}{n}$
\begin{align*}
f(1)=f(m)&=f(n\frac{m}{n})\\
&=nf(\frac{m}{n})\\
f(\frac{m}{n})&=\frac{m}{n}f(1)
\end{align*}

5. $\forall x\in\mathbb{R}$ da $\mathbb{Q}$ in $\mathbb{R}$ dicht ist,

$\exists x_n\subseteq\mathbb{Q}$ sodass $\lim_{n\rightarrow\infty}x_n=x$,

$f(x)=f(\lim_{n\rightarrow\infty}x_n)=\lim_{n\rightarrow\infty}f(x_n)$, $\lim_{n\rightarrow\infty}x_nf(1)=xf(1)$\\

$f(x)=cx$ mit $c\in\mathbb{R}$

\newpage

\subsection{tut}

\begin{definition}[Stetigkeit]
$\lim_{x\rightarrow a}f(x)=f(a)$ heißt:\\

$\forall(a_n)\subseteq D$, $f(a_n)$ konvegiert gegen $f(a)$
\end{definition}

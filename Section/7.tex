\section{7. Übungsblatt}

\subsection{Aufgabe 7.1}

\subsection{Aufgabe 7.2}

\subsection{Aufgabe 7.3}

\subsection{Aufgabe 7.4}

\subsection{Aufgabe 7.5}

\newpage

\subsection{Aufgabe 7.6(H)}

Wir beweisen zuerst die folgende Gleichung
\begin{equation*}
\underset{\geq\sqrt{a}}{\underbrace{(\sqrt{a_n}+\sqrt{a})}}\cdot|\sqrt{a_n}-\sqrt{a}|=|a_n-a|
\end{equation*}
\begin{proof}
\begin{equation*}
(\sqrt{a_n}+\sqrt{a})\cdot|\sqrt{a_n}-\sqrt{a}|=\left\{\begin{array}{lcl}
(\sqrt{a_n}>\sqrt{a}) & (\sqrt{a_n}+\sqrt{a})\cdot(\sqrt{a_n}-\sqrt{a})=(a_n-a)=|a_n-a|\\
(\sqrt{a_n}<\sqrt{a}) & (\sqrt{a_n}+\sqrt{a})\cdot(\sqrt{a}-\sqrt{a_n})=-(a_n-a)=|a_n-a|
\end{array}\right.
\end{equation*}

$a_n\geq0$ $\Leftrightarrow$ $\sqrt{a_n}\geq0$

$\sqrt{a_n}\geq0$ $\Leftrightarrow$ $\sqrt{a_n}+\sqrt{a}\geq\sqrt{a}$

\end{proof}

Sei $\lim_{n\rightarrow\infty}a_n=a$

$\Leftrightarrow$ $\forall\varepsilon\in\mathbb{R}_{>0}$, $\exists N\in\mathbb{N}$, $\forall n\in\mathbb{N}_{>N}$, $|a_n-a|<\varepsilon$\\

dann

\begin{align*}
(\sqrt{a_n}+\sqrt{a})\cdot|\sqrt{a_n}-\sqrt{a}|&=|a_n-a|\\
|\sqrt{a_n}-\sqrt{a}|&=\frac{|a_n-a|}{(\sqrt{a_n}+\sqrt{a})}<\frac{\varepsilon}{\sqrt{a}}
\end{align*}

Sei $\tilde{\varepsilon}=\frac{\varepsilon}{\sqrt{a}}\in\mathbb{R}_{>0}$, dann $|\sqrt{a_n}-\sqrt{a}|<\tilde{\varepsilon}$

$\Leftrightarrow$ $\lim_{n\rightarrow\infty}\sqrt{a_n}=\sqrt{a}$

\newpage

\subsection{Aufgabe 7.7(H)}
$ $\newline

Wir beweisen nun mittels Vollständige Induktion, dass $a_n$ monoton wachsend ist.
\begin{proof}
$ $\newline

Sei $a_{n+1}=\sqrt{c+a_n}$\\

IA:

Sei $n=1$, dann gilt
\begin{align*}
\frac{a_2}{a_1}&=\frac{\sqrt{c+\sqrt{c}}}{\sqrt{c}}\\
\frac{a_2^2}{a_1^2}&=\frac{c+\sqrt{c}}{c}\overset{(c>0)}{>}1\\
\overset{(c>0)}{\Leftrightarrow}\frac{a_2}{a_1}&>1
\end{align*}\\

IV:

\begin{equation*}
\frac{a_{n+1}}{a_n}>1\hspace{1em}\Leftrightarrow\hspace{1em}\frac{\sqrt{c+a_n}}{a_n}>1
\end{equation*}\\

IS:

Wir setzen $\tilde{n}=n+1$ ein,
\begin{align*}
\frac{a_{\tilde{n}+1}}{a_{\tilde{n}}}=\frac{a_{(n+1)+1}}{a_{n+1}}
&=\frac{\sqrt{c+a_{n+1}}}{\sqrt{c+a_n}}\\
\frac{(a_{(n+1)+1})^2}{a_{n+1}^2}
&=\frac{c+\sqrt{c+a_n}}{c+a_n}\overunderset{\mathbf{IV}}{(c>0)}{>}1\\
\overset{(c>0)}{\Leftrightarrow}\frac{a_{\tilde{n}+1}}{a_{\tilde{n}}}&>1
\end{align*}
\end{proof}

\newpage

Wir beweisen nun, dass $a_n$ von oben beschränkt ist.
\begin{proof}
\begin{align*}
a_{n+1}&=\sqrt{c+a_n}\\
a_n&=\sqrt{c+a_{n-1}}\\
\Rightarrow\lim_{n\rightarrow\infty}a_n&=\sqrt{\lim_{n\rightarrow\infty}(c+a_{n-1})}\\
&=\sqrt{c+\lim_{n\rightarrow\infty}a_{n-1}}
\end{align*}

Sei $\lim_{n\rightarrow\infty}a_n=a=\lim_{n\rightarrow\infty}a_{n-1}$, dann
\begin{align*}
a&=\sqrt{c+a}\\
a^2&=c+a\\
a^2-a-c&=0\\
\Leftrightarrow a&=\left\{\begin{array}{lcl}
\frac{1+\sqrt{1+4c}}{2}\overset{(c>0)}{>}1 \\
\frac{1-\sqrt{1+4c}}{2}\overset{(c>0)}{<}0
\end{array}\right.
\end{align*}
Wir haben schon gezeigt, dass $a_n>0$ monoton wachsend ist, dann  muss $a>0$ sein. Deshalb haben wir $a=\frac{1+\sqrt{1+4c}}{2}$. Dann ist $\lim_{n\rightarrow\infty}a_n=a=\frac{1+\sqrt{1+4c}}{2}$

\end{proof}

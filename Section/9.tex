\section{9. Übungsblatt}

\subsection{Aufgabe 9.1}

\newpage

\subsection{Aufgabe 9.2}

\begin{proof}
$ $\newline

\begin{align*}
(1+\frac{1}{n})^n
&=\sum_{k=0}^n\begin{pmatrix}
n \\
k
\end{pmatrix}
(\frac{1}{n})^k\\
&=\sum_{k=0}^n\frac{n!}{k!(n-k)!}\frac{1}{n^k}\\
\end{align*}

Term:
\begin{equation*}
\frac{1}{n^k}\frac{n!}{(n-k)!}=(1-\frac{k-1}{n})(1-\frac{k-2}{n})\cdots
\end{equation*}

\begin{align*}
&=\sum_{k=0}^n\frac{1}{k!}(1-\frac{1}{n})(1-\frac{2}{n})\cdots(1-\frac{k-1}{n})\\
&=\sum_{k=0}^m\frac{1}{k!}(1-\frac{1}{n})(1-\frac{2}{n})\cdots(1-\frac{k-1}{n})\\
(n>m)&=1+1+\frac{1}{2}(1-\frac{1}{n})+\cdots+\frac{1}{m!}(1-\frac{1}{n})+\cdots+(1-\frac{m-1}{n})
\end{align*}

\end{proof}

\newpage

\subsection{Aufgabe 9.3}

\newpage

\subsection{Aufgabe 9.4}

\begin{equation*}
s=\sum_{n=1}^{\infty}\frac{(-1)^{n-1}}{n}=1-\frac{1}{2}+\frac{1}{3}-\frac{1}{4}+\ldots
\end{equation*}

konvergent, aber nicht abs. konv.\\

$s^+$ und $s^-$ sind Umordnung  von $s$

\begin{equation*}
s^+_{3n}=\sum_{k=1}^{2n}\frac{1}{2k-1}-\sum_{k=1}^{n}\frac{1}{2k}
\end{equation*}
\begin{equation*}
s^-_{3n}=\sum_{k=1}^{n}\frac{1}{2k-1}-\sum_{k=1}^{2n}\frac{1}{2k}
\end{equation*}

wir zerlegen auch $s$ zur $s_{2n}$
\begin{equation*}
s_{2n}=\sum_{k=1}^{n}\frac{1}{2k-1}-\sum_{k=1}^{n}\frac{1}{2k}
\end{equation*}

mit $n=2n$
\begin{equation*}
s_{4n}=s_{2(2n)}=\sum_{k=1}^{2n}\frac{1}{2k-1}-\sum_{k=1}^{2n}\frac{1}{2k}
\end{equation*}

\newpage

\subsection{Aufgabe 9.5(H)}
\begin{proof}
$ $\newline

z.z. ist
\begin{equation*}
\Bigg(\sum_{k=0}^\infty q^k\Bigg)^2=\sum_{n=0}^\infty(n+1)\cdot q^n
\end{equation*}

Def. Cauchy-Produkt:
\begin{equation*}
\Bigg(\sum_{i=0}^\infty a_i\Bigg)\cdot\Bigg(\sum_{j=0}^\infty b_j\Bigg)=\sum_{k=0}^\infty\sum_{l=0}^k a_lb_{k-l}
\end{equation*}

dann gilt
\begin{align*}
\Bigg(\sum_{i=0}^\infty q^i\Bigg)\cdot\Bigg(\sum_{j=0}^\infty q^j\Bigg)
&=\sum_{k=0}^\infty\Bigg(\sum_{l=0}^k q^lq^{k-l}\Bigg)\\
&=\sum_{k=0}^\infty\Bigg((q^0q^{k-0})+(q^1q^{k-1})+(q^2q^{k-2})+\ldots+(q^kq^{k-k})\Bigg)\\
&=\sum_{k=0}^\infty(k+1)\cdot q^k
\end{align*}
\end{proof}

\newpage

\subsection{Aufgabe 9.6(H)}
\begin{proof}
$ $\newline

Gegeben sei $a_n:=b_n:=\frac{(-1)^n}{\sqrt{n+1}}$
\begin{equation*}
\sum_{n=0}^\infty a_n=\sum_{n=0}^\infty\frac{(-1)^n}{\sqrt{n+1}}=\sum_{n=0}^\infty\frac{1}{\sqrt{n+1}}(-1)^n
\end{equation*}
mit
\begin{equation*}
\lim_{n\rightarrow\infty}\frac{1}{\sqrt{n+1}}=0
\end{equation*}
$\Rightarrow$ $\sum_{n=0}^\infty a_n$ konvergiert.
\end{proof}

\begin{proof}
$ $\newline

Mit Cauchy-Produkt
\begin{align*}
\Bigg(\sum_{n=0}^\infty a_n\Bigg)^2=\Bigg(\sum_{n=0}^\infty\frac{(-1)^n}{\sqrt{n+1}}\Bigg)^2
&=\sum_{n=0}^\infty\Bigg(\sum_{k=0}^n\frac{(-1)^{n-k}}{\sqrt{(n-k)+1}}\frac{(-1)^k}{\sqrt{k+1}}\Bigg)\\
&=\sum_{n=0}^\infty\Bigg(\sum_{k=0}^n\frac{1}{\sqrt{(n-k)+1}}\frac{1}{\sqrt{k+1}}\Bigg)(-1)^n
\end{align*}
mit
\begin{equation*}
\frac{1}{\sqrt{(n-k)+1}}\geq\frac{1}{\sqrt{n+1}}\hspace{1em}\mbox{und}\hspace{1em}\frac{1}{\sqrt{k+1}}\geq\frac{1}{\sqrt{n+1}}
\end{equation*}
dann gilt
\begin{equation*}
\frac{1}{\sqrt{(n-k)+1}}\frac{1}{\sqrt{k+1}}\geq\frac{1}{n+1}
\end{equation*}
und
\begin{equation*}
\sum_{n=0}^\infty\Bigg(\sum_{k=0}^n\frac{1}{\sqrt{(n-k)+1}}\frac{1}{\sqrt{k+1}}\Bigg)(-1)^n\geq\sum_{n=0}^\infty\Bigg(\sum_{k=0}^n\frac{1}{k+1}\Bigg)(-1)^n
\end{equation*}
mit
\begin{equation*}
\lim_{n\rightarrow\infty}\sum_{k=0}^n\frac{1}{k+1}=\infty\ (\mbox{divergent})
\end{equation*}
dann
\begin{equation*}
\sum_{n=0}^\infty\Bigg(\sum_{k=0}^n\frac{1}{\sqrt{(n-k)+1}}\frac{1}{\sqrt{k+1}}\Bigg)(-1)^n
\end{equation*}
divergent
\end{proof}

\newpage

\subsection{Tutorium}

Exponentialreihe:

$\exp(x):=\sum_{n=0}^{\infty}\frac{x}{n!}$ insb. $e=\sum_{n=0}^{\infty}\frac{1}{n!}$

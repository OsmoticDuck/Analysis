\section{12. Übungsblatt}

\subsection{12.1}

\subsection{12.2}

\paragraph{(c)}
$ $\newline



\newpage

\subsection{Aufgabe 12.3}

(L-Stedigkeit schon gegeben)

\begin{proof}
$ $\newline

Für $\delta:=\frac{\varepsilon}{L}$, $\forall x_1,x_2\in D$, $|x_1-x_2|<\delta$, gilt
\begin{equation*}
|f(x_1)-f(x_2)|\leq L|x_1-x_2|<L\delta=\varepsilon
\end{equation*}
\end{proof}

\subsection{Aufgabe 12.4}

\paragraph{(a)}
$ $\newline

\begin{proof}
$ $\newline

Verneinung der Def.

$\exists \varepsilon>0$, $\forall\delta>0$ $\exists x_1,x_2\in D$ gilt
\begin{equation*}
|x_1-x_2|<\delta,|f(x_1)-f(x_2)|\geq\varepsilon
\end{equation*}

dann, sei $\varepsilon_0=\frac{1}{2}$, $x_1=\frac{1}{\delta}$ und $x_2=x_1+\frac{\delta}{2}$

es gilt
\begin{equation*}
|x_1-x_2|=\frac{\delta}{2}
\end{equation*}

und
\begin{equation*}
|f(x_1)-f(x_2)|=|x_1^2-x_2^2|=|\frac{1}{\delta^2}-\frac{1}{\delta^2}+\delta x_1+\frac{\delta^2}{4}|
\end{equation*}

dazu $\delta x_1=1>\frac{1}{2}$

\end{proof}

\paragraph{(c)}
$ $\newline

z.B. $L=1$

\begin{proof}
\begin{align*}
|\sqrt{x_1}-\sqrt{x_2}|=\frac{|x_1-x_2|}{|\sqrt{x_1}+\sqrt{x_2}|}=\frac{1}{2}|x_1-x_2|<1
\end{align*}
\end{proof}

\newpage

\paragraph{(d)}
$ $\newline

Hint: $[0,\infty]=[0,2]\cup[1,\infty]$

in (c) schon bewiesen $[1,2]$ glm. stetig, dann $\Rightarrow \square$

\subsection{12.5}

\subsection{12.6}

\subsection{12.7}

\newpage

\subsection{Tutorium}

\begin{definition}[L-stetig]
$\exists L\geq0$ $\forall x_1,x_2\in D$ gilt

\begin{equation*}
|f(x_1)-f(x_2)|\leq L|x_1-x_2|
\end{equation*}

\end{definition}

\section{16. Übungsblatt}

\paragraph{Aufgabe 16.3(H)}

\begin{enumerate}

\item[(a)]

$f(x):=x\cdot e^x$
\begin{align*}
f(x)=f^{(0)}&=x\cdot e^x\\
f^{(1)}&=e^x+x\cdot e^x\\
f^{(2)}&=e^x+e^x+x\cdot e^x
\end{align*}

$erraten$:

\begin{equation}
f^{(n)}=n\cdot e^x+x\cdot e^x
\end{equation}

\begin{proof}
$ $\newline

$\mathbf{IA}$: Sei $n=0$, es gilt $f(x)=f^{(0)}=x\cdot e^x$\\

$\mathbf{IV}$:
\begin{equation}
f^{(n)}=n\cdot e^x+x\cdot e^x
\end{equation}\\

$\mathbf{IS}$: Sei $\tilde{n}=n+1$
\begin{align}
f^{(\tilde{n})}&=f^{(n+1)}\\
&=(f^{(n)})'\\
&=n\cdot e^x+e^x+x\cdot e^x\\
&=(n+1)\cdot e^x+x\cdot e^x
\end{align}
\end{proof}

$g(x):=\sin^2x$
\begin{align*}
f(x)=f^{(0)}&=\sin^2x\\
f^{(1)}&=2\sin x\cdot\cos x=\sin 2x\\
f^{(2)}&=2\cos 2x\\
f^{(3)}&=-4\sin 2x\\
f^{(4)}&=-8\cos 2x\\
f^{(5)}&=16\sin 2x\\
f^{(6)}&=32\cos 2x
\end{align*}

$erraten$:

$n=2k-1$, $k\in\mathbb{N}_{>0}$:
\begin{equation}
f^{(n)}=-1^{\frac{n-1}{2}}\cdot 2^{n-1}\cdot\sin 2x
\end{equation}

$n=2k$, $k\in\mathbb{N}_{>0}$:
\begin{equation}
f^{(n)}=-1^{\frac{n-2}{2}}\cdot 2^{n-1}\cdot\cos 2x
\end{equation}\\

Induktion für $n=2k-1$, $k\in\mathbb{N}_{>0}$:
\begin{proof}
$ $\newline

$\mathbf{IA}$: Sei $n=1$, es gilt $f^{(1)}=\sin 2x$\\

$\mathbf{IV}$:
\begin{equation}
f^{(n)}=-1^{\frac{n-1}{2}}\cdot 2^{n-1}\cdot\sin 2x
\end{equation}\\

$\mathbf{IS}$: Sei $\tilde{n}=n+2$
\begin{align}
f^{(\tilde{n})}&=f^{(n+2)}\\
&=(f^{(n)})''\\
&=-1^{\frac{n-1}{2}}\cdot 2^{n-1}\cdot 2^2\cdot -1\cdot \sin 2x\\
&=-1^{\frac{(n+2)-1}{2}}\cdot 2^{(n+2)-1}\cdot \sin 2x
\end{align}
\end{proof}

Induktion für $n=2k$, $k\in\mathbb{N}_{>0}$:
\begin{proof}
$ $\newline

$\mathbf{IA}$: Sei $n=2$, es gilt $f^{(2)}=2\cos 2x$\\

$\mathbf{IV}$:
\begin{equation}
f^{(n)}=-1^{\frac{n-2}{2}}\cdot 2^{n-1}\cdot\cos 2x
\end{equation}\\

$\mathbf{IS}$: Sei $\tilde{n}=n+2$
\begin{align}
f^{(\tilde{n})}&=f^{(n+2)}\\
&=(f^{(n)})''\\
&=-1^{\frac{n-2}{2}}\cdot 2^{n-1}\cdot 2^2\cdot -1\cdot \cos 2x\\
&=-1^{\frac{(n+2)-2}{2}}\cdot 2^{(n+2)-1}\cdot \cos 2x
\end{align}
\end{proof}

\newpage

\item[(b)]

\begin{align*}
h=h^{(0)}&=x^2\cdot e^x\\
h^{(1)}&=(0+2x+x^2)e^x\\
h^{(2)}&=(2+4x+x^2)e^x\\
h^{(3)}&=(6+6x+x^2)e^x\\
h^{(4)}&=(12+8x+x^2)e^x\\
h^{(5)}&=(20+10x+x^2)e^x\\
h^{(6)}&=(30+12x+x^2)e^x\\
&\hspace{6pt}\vdots\\
h^{(1000)}&=(999000+2000x+x^2)e^x
\end{align*}

$erraten$:
\begin{equation*}
h^{(n)}=((n^2-n)+2nx+x^2)e^x
\end{equation*}

\begin{proof}
$ $\newline

$\mathbf{IA}$: Sei $n=0$, es gilt $h^{(0)}=x^2\cdot e^x$\\

$\mathbf{IV}$:
$erraten$:
\begin{equation*}
h^{(n)}=((n^2-n)+2nx+x^2)e^x
\end{equation*}\\

$\mathbf{IS}$: Sei $\tilde{n}=n+1$
\begin{align}
h^{(\tilde{n})}=h^{(n+1)}&=(h^{(n)})'\\
&=(2n+2x+(n^2-n)+2nx+x^2)e^x\\
&=(((n+1)^2-(n+1))+2(n+1)x+x^2)e^x
\end{align}
\end{proof}

\end{enumerate}
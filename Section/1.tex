\section{1. Übungsblatt: $Aussagen\ und\ Quantoren$}

\subsection{Aufgabe 1.1}

\paragraph{(a)}
\begin{proof}
$ $\newline
Sei $p$ $wahr$:
\begin{equation}
p\ wahr\Rightarrow\neg p\ falsch
\end{equation}
\begin{equation}
\Rightarrow (p\wedge\neg p)\ falsch
\end{equation}
\begin{equation}
\Rightarrow \neg(p\wedge\neg p)\ wahr
\end{equation}\\
Sei $p$ $falsch$:
\begin{equation}
p\ falsch\Rightarrow\neg p\ wahr
\end{equation}
\begin{equation}
\Rightarrow (p\wedge\neg p)\ falsch
\end{equation}
\begin{equation}
\Rightarrow \neg(p\wedge\neg p)\ wahr
\end{equation}\\
$Folgerung:$ Diese Formel ist Tautologie.
\end{proof}

\paragraph{(b)}
\begin{proof}
$ $\newline
\begin{center}
\begin{tabular}{||c|c||c|c|c|c|c|c||}
\hline
$p$ & $q$ & $\neg p$ & $\neg q$ & $p\wedge q$ & $\neg(p\wedge q)$ & $(\neg p)\vee (\neg q)$ & $\neg(p\wedge q)\Leftrightarrow((\neg p)\vee (\neg q))$ \\
\hline
\hline
$w$ & $w$ & $f$ & $f$ & $w$ & $f$ & $f$ & $w$ \\
%\hline
$w$ & $f$ & $f$ & $w$ & $f$ & $w$ & $w$ & $w$ \\
%\hline
$f$ & $w$ & $w$ & $f$ & $f$ & $w$ & $w$ & $w$ \\
%\hline
$f$ & $f$ & $w$ & $w$ & $f$ & $w$ & $w$ & $w$ \\
\hline
\end{tabular}
\end{center}
$Folgerung:$ Diese Formel ist Tautologie.
\end{proof}

\paragraph{(c)}
\begin{proof}
$ $\newline
\begin{center}
\begin{tabular}{||c|c||c|c|c|c|c||}
\hline
$p$ & $q$ & $\neg p$ & $\neg q$ & $p\Rightarrow q$ & $(p\Rightarrow q)\wedge (\neg q)$ & $((p\Rightarrow q)\wedge (\neg q))\Rightarrow (\neg p)$ \\
\hline
\hline
$w$ & $w$ & $f$ & $f$ & $w$ & $f$ & $w$ \\
%\hline
$w$ & $f$ & $f$ & $w$ & $f$ & $f$ & $w$ \\
%\hline
$f$ & $w$ & $w$ & $f$ & $w$ & $f$ & $w$ \\
%\hline
$f$ & $f$ & $w$ & $w$ & $w$ & $w$ & $w$ \\
\hline
\end{tabular}
\end{center}
$Folgerung:$ Diese Formel ist Tautologie.
\end{proof}

\newpage

\paragraph{(d)}
\begin{proof}
$ $\newline
\begin{center}
\begin{tabular}{||c|c|c||c|c|c|c|c|c||}
\hline
$p$ & $q$ & $r$ & $q\wedge r$ & $p\vee q$ & $p\vee r$ & $p\vee (q\wedge r)$ & $(p\vee q)\wedge (p\vee r)$ & $(p\vee (q\wedge r))\Leftrightarrow ((p\vee q)\wedge (p\vee r))$ \\
\hline
\hline
$w$ & $w$ & $w$ & $w$ & $w$ & $w$ & $w$ & $w$ & $w$ \\

$w$ & $w$ & $f$ & $f$ & $w$ & $w$ & $w$ & $w$ & $w$ \\

$w$ & $f$ & $w$ & $f$ & $w$ & $w$ & $w$ & $w$ & $w$ \\

$w$ & $f$ & $f$ & $f$ & $w$ & $w$ & $w$ & $w$ & $w$ \\

$f$ & $w$ & $w$ & $w$ & $w$ & $w$ & $w$ & $w$ & $w$ \\

$f$ & $w$ & $f$ & $f$ & $w$ & $f$ & $f$ & $f$ & $w$ \\

$f$ & $f$ & $w$ & $f$ & $f$ & $w$ & $f$ & $f$ & $w$ \\

$f$ & $f$ & $f$ & $f$ & $f$ & $f$ & $f$ & $f$ & $w$ \\
\hline
\end{tabular}
\end{center}
$Folgerung:$ Diese Formel ist Tautologie.
\end{proof}

\paragraph{(e)}
$ $\newline
Die Negation von Satz "Satz vom Widerspruch" ist einfach eine Kontradiktion:
\begin{equation*}
\neg (\neg (p\wedge \neg p))\Leftrightarrow p\wedge \neg p\ \ (stets\ falsch)
\end{equation*}

\newpage

\subsection{Aufgabe 1.2}

\paragraph{}
\begin{solution}
$ $\newline
Wir vereinfachen nun die Formel $(\neg ((p\Rightarrow (q\Rightarrow p))\wedge p)\vee (q\Rightarrow q))\Rightarrow p$ \\
Nach Wahrheitstabelle gilt es:
\begin{equation*}
(p\Rightarrow (q\Rightarrow p))\Leftrightarrow p,\ (q\Rightarrow q)\Leftrightarrow q
\end{equation*}
Damit vereinfachen wir die Formel zu:
\begin{equation}
(\neg (p\wedge p)\vee q)\Rightarrow p
\end{equation}
Nach Wahrheitstabelle gilt es:
\begin{equation*}
(p\vee p)\Leftrightarrow p
\end{equation*}
Damit vereinfachen wir die Formel zu:
\begin{equation}
(\neg p\vee q)\Rightarrow p
\end{equation}
Nach Priorität von Verknüpfungen können wir noch vereinfachen:
\begin{equation}
\neg p\vee q\Rightarrow p
\end{equation}
\end{solution}
\begin{remark}
$(a\Rightarrow b)\Leftrightarrow(\neg a\wedge b)$
\end{remark}

\newpage

\subsection{Aufgabe 1.3}

\paragraph{(a)}
\begin{solution}
$ $\newline
$(x\in\mathbb{R})\wedge(x^2=1)\Leftrightarrow (x=1)\wedge(x=-1)$
\end{solution}
\textbf{Negation}\\
$\forall x\in\mathbb{R}$ und $x^2=1$ $\Rightarrow(x=1)\vee(x=-1)$

\paragraph{(b)}
\begin{solution}
$ $\newline
$\neg(\exists x\in\mathbb{R})\Rightarrow(x^2=-1)$
\end{solution}
\textbf{Negation}\\
$(\exists x\in\mathbb{R})\Rightarrow(x^2=-1)$
\begin{remark}
Es gilt auch $\nexists x\in\mathbb{R}$
\end{remark}

\paragraph{(c)}
\begin{solution}
$ $\newline
$(x\in\mathbb{N})\wedge((\frac{x}{6}\in\mathbb{Z})\vee((\frac{x}{4}\in\mathbb{Z})\wedge(\frac{x}{9}\in\mathbb{Z})))\Rightarrow(\frac{x}{2}\in\mathbb{Z})\wedge(\frac{x}{3}\in\mathbb{Z})$
\end{solution}
\textbf{Negation}\\
$\exists n\in\mathbb{N}:\ (6|n)\vee((4|n)\wedge(9|n))\Rightarrow(2\nmid n)\vee(3\nmid n)$
\begin{remark}
$(a|b)$ bedeutet: $b$ durch $a$ teilbar
\end{remark}

\paragraph{(d)}
\begin{solution}
$ $\newline
$(\forall x\in\mathbb{N})\Rightarrow(\exists y\in\mathbb{P})\wedge(y>x)$
\end{solution}
\textbf{Negation}\\
$\exists n\in\mathbb{N}\ \forall p\in\mathbb{P}:\ p\leq n$

\paragraph{(e)}
\begin{solution}
$ $\newline
$(M\subset\mathbb{Z})\Rightarrow(\exists!k\in M)\wedge(\forall g\in M\setminus k)\wedge(g>k)$
\end{solution}
\textbf{Negation}\\
$\exists M\subset\mathbb{N}:\ \forall m\in\ M\ \exists n\in\ M\setminus\{m\}:\ m\geq n$\\
oder\\
$\exists m_1,\ m_2\in\ M,\ m_1\neq m_2:\ m_1<n,\ m_2<n$

\newpage

\subsection{Aufgabe 1.4}
Wir nennen: $G,\ H,\ N,\ D,\ R,\ Z,\ S$\\
\begin{align}
H\Rightarrow \neg G\\
G\wedge \neg R\Rightarrow D\\
\neg N\wedge \neg G\Rightarrow Z\vee D\\
\neg S\Rightarrow\neg D\\
G\Rightarrow S\veebar H\\
R\Rightarrow N\veebar G\\
Z\Rightarrow R\\
S\Rightarrow N\\
N\Rightarrow\neg R\wedge G
\end{align}
$Folgerung:$ Er ist krank.\\
\null
Nehmen wir an, dass:\\
\begin{equation*}
\neg N\wedge \neg G
\end{equation*}
gilt:\\
\begin{equation*}
\Rightarrow^{(3)} Z\vee G
\end{equation*}
falls:\\
\begin{align*}
Z=1\Rightarrow^{(1)}R\Rightarrow^{(6)}N\veebar G\\
D=1\Rightarrow^{(4)}S\Rightarrow^{(8)}N\\
N=1\Rightarrow^{(9)}\neg R\wedge G &\Rightarrow^{(7)}\neg Z\\
&\Rightarrow^{(2)}S\veebar H\\
&\Rightarrow^{(1)}\neg	H
\end{align*}
\begin{equation*}
N\wedge S\wedge\neg H\wedge\neg R\wedge G\wedge\neg Z\wedge D
\end{equation*}
Falls:
\begin{equation*}
G\wedge\neg N
\end{equation*}
gilt es:
\begin{align*}
S\veebar H\\
\end{align*}

\newpage

\subsection{Aufgabe 1.5(H)}

\paragraph{(a)}
\begin{proof}
$ $\newline
\begin{center}
\begin{tabular}{||c|c|c||c|c|c|c|c|c||}
\hline
$p$ & $q$ & $r$ & $p\vee q$ & $p\wedge r$ & $q\wedge r$ & $(p\vee q)\wedge r$ & $(p\wedge r)\vee(q\wedge r)$ & $((p\vee q)\wedge r)\Leftrightarrow((p\wedge r)\vee(q\wedge r))$ \\
\hline
\hline
$w$ & $w$ & $w$ & $w$ & $w$ & $w$ & $w$ & $w$ & $w$ \\
$w$ & $w$ & $f$ & $w$ & $f$ & $f$ & $f$ & $f$ & $w$ \\
$w$ & $f$ & $w$ & $w$ & $w$ & $f$ & $w$ & $w$ & $w$ \\
$w$ & $f$ & $f$ & $w$ & $f$ & $f$ & $f$ & $f$ & $w$ \\
$f$ & $w$ & $w$ & $w$ & $f$ & $w$ & $w$ & $w$ & $w$ \\
$f$ & $w$ & $f$ & $w$ & $f$ & $f$ & $f$ & $f$ & $w$ \\
$f$ & $f$ & $w$ & $f$ & $f$ & $f$ & $f$ & $f$ & $w$ \\
$f$ & $f$ & $f$ & $f$ & $f$ & $f$ & $f$ & $f$ & $w$ \\
\hline
\end{tabular}
\end{center}
$Folgerung:$ Diese Formel ist Tautologie.
\end{proof}

\paragraph{(b)}
\begin{proof}
$ $\newline
\begin{center}
\begin{tabular}{||c|c||c|c|c|c||}
\hline
$p$ & $q$ & $\neg p$ & $p\Rightarrow q$ & $(\neg p)\vee q$ & $(p\Rightarrow q)\Leftrightarrow((\neg p)\vee q)$ \\
\hline
\hline
$w$ & $w$ & $f$ & $w$ & $w$ & $w$ \\
$w$ & $f$ & $f$ & $f$ & $f$ & $w$ \\
$f$ & $w$ & $w$ & $w$ & $w$ & $w$ \\
$f$ & $f$ & $w$ & $w$ & $w$ & $w$ \\
\hline
\end{tabular}
\end{center}
$Folgerung:$ Diese Formel ist Tautologie.
\end{proof}

\paragraph{(c)}
\begin{proof}
$ $\newline
\begin{center}
\begin{tabular}{||c|c||c|c|c|c|c||}
\hline
$p$ & $q$ & $\neg p$ & $\neg q$ & $p\Rightarrow q$ & $(\neg q)\Rightarrow(\neg p)$ & $(p\Rightarrow q)\Leftrightarrow((\neg q)\Rightarrow(\neg p))$ \\
\hline
\hline
$w$ & $w$ & $f$ & $f$ & $w$ & $w$ & $w$ \\
$w$ & $f$ & $f$ & $w$ & $f$ & $f$ & $w$ \\
$f$ & $w$ & $w$ & $f$ & $w$ & $w$ & $w$ \\
$f$ & $f$ & $w$ & $w$ & $w$ & $w$ & $w$ \\
\hline
\end{tabular}
\end{center}
$Folgerung:$ Diese Formel ist Tautologie.
\end{proof}

\newpage

\subsection{Aufgabe 1.6(H)}

\paragraph{Negation von $\varphi$}
\begin{proof}
\begin{equation*}
(\varphi|\varphi)\Leftrightarrow(\neg(\varphi\wedge\varphi))\Leftrightarrow(\neg\varphi)
\end{equation*}
\end{proof}

\paragraph{Konjunktion von $\varphi$ und $\psi$}
\begin{proof}
\begin{equation*}
((\varphi|\psi)|(\varphi|\psi))\Leftrightarrow(\neg((\neg(\varphi\wedge\psi))\wedge(\neg(\varphi\wedge\psi))))\Leftrightarrow(\neg(\neg(\varphi\wedge\psi)))\Leftrightarrow(\varphi\wedge\psi)
\end{equation*}
\end{proof}

\paragraph{Disjunktion von $\varphi$ und $\psi$}
\begin{proof}
\begin{equation*}
((\varphi|\varphi)|(\psi|\psi))\Leftrightarrow((\neg\varphi)|(\neg\psi))\Leftrightarrow(\neg((\neg\varphi)\wedge(\neg\psi)))\Leftrightarrow(\neg(\neg(\varphi\vee\psi)))\Leftrightarrow(\varphi\vee\psi)
\end{equation*}
\end{proof}

\newpage

\subsection{Tutorium}

\paragraph{1. Aussagen}
$ $\newline
Wir nehmen $wahr\ 1$ und $falsch\ 0$\\
Sei $a$ ein Aussagen, dann gilt:
\begin{equation*}
a=1\ oder\ 0
\end{equation*}

\paragraph{2.Verknüpfungen}
\begin{align*}
\neg\ Negation\\
\wedge\ Konjugtion\\
\vee\ Disjunktion\\
\veebar\ XOR\ \triangleq\ "entweder\ldots oder\ldots"
\end{align*}

\paragraph{3.Wahrheitstabelle}
\begin{exmp}
$ $\newline
\begin{center}
\begin{tabular}{|c||c|}
\hline
$a$ & $\neg a$ \\
\hline
$0$ & $1$ \\
$1$ & $0$ \\
\hline
\end{tabular}
\end{center}
\end{exmp}

\paragraph{4.Quantor}
\begin{center}
$\forall$ "für alle"\\
$\exists$ "existiert"\\
$\exists !$ "existiert genau eine"
\end{center}

\newpage

\begin{remark}
$ $\newline
\textbf{Tautologie}: Immer $wahr$\\
\begin{equation*}
p\vee\neg p=1
\end{equation*}
\null
\end{remark}

\section{8. Übungsblatt}

\subsection{Aufgabe 8.1}

\begin{proof}
$ $\newline

\end{proof}

\subsection{Aufgabe 8.2}

c) geo. Reihe mit

\newpage

\subsection{Aufgabe 8.3}
$ $\newline

Maj. Krit. : Seien $(a_n)_n,(b_n)_n\subseteq\mathbb{R}$, $\forall n\in\mathbb{N}$: $b_n>0$

Falls $\exists N\in\mathbb{N}$, $\exists c\in\mathbb{R}$, $\forall n\in\mathbb{N}_{>N}$: $|a_n|\leq c\cdot b_n$

und falls $\sum_{1}^{\infty}b_n$ konv. dann konv. $\sum_{1}^{\infty}$ abs.

\begin{proof}
$ $\newline

$(a_n)_n\in\mathbb{R}$ und $\exists\theta\in(0,1)$ und $\exists N\in\mathbb{N}$

sodass: $\forall n\geq N$ gilt $\sqrt[n]{|a_n|}<\theta$

da $\sum_{n=1}^{\infty}\theta^n$ konv. (geom. Reihe) $\Leftrightarrow$ $|a_n|<\theta^n$

nach Maj. Krit. konv. $\sum a_n$ abs.
\end{proof}

\newpage

\subsection{Aufgabe 8.4}

\paragraph{(a)}
$ $\newline

$a_n=\frac{n!}{n^n}$

Quotient-Krit.

\begin{align*}
\Big|\frac{a_{n+1}}{a_n}\Big|
&=\Big|\frac{(n+1)!}{(n+1)^{n+1}}\Big|\\
&=\Big|\frac{n^n}{(n+1)^n}\Big|\\
()^{-1}&=\underset{(n\rightarrow0)\rightarrow e}{\underbrace{\Bigg(\Big(\frac{n+1}{n}\Big)^n\Bigg)^{-1}}}
\end{align*}

$e^x:=\lim_{n\rightarrow\infty}(1+\frac{x}{n})^n$

$\Rightarrow$ $|\frac{a_{n+1}}{a_n}|=((1+\frac{1}{n})^n)^{-1}\overset{n\rightarrow\infty}{\longrightarrow}e^{-1}$\\

Sei nun $\tilde{\epsilon}=1-e^{-1}$ dann: $\underset{\theta}{\underbrace{e^{-1}+\frac{\tilde{\epsilon}}{2}}}<1$

$e^{-1}+\frac{\tilde{\epsilon}}{2}=e^{-1}+\frac{1-e^{-1}}{2}=\frac{1+e^{-1}}{2}<1$

dann $\lim_{n\rightarrow\infty}|\frac{a_{n+1}}{a_n}|=e^{-1}$ for $\epsilon_i=\frac{\tilde{\epsilon}}{2}$ $\exists N\in\mathbb{N}$ sind

$\forall n\in\mathbb{N}_{>N}$: $|\frac{a_{n+1}}{a_n}-e^{-1}|<\frac{\tilde{\epsilon}}{2}$

$\Rightarrow$ $\frac{a_{n+1}}{a_n}<e^{-1}+\frac{\tilde{\epsilon}}{2}$

\paragraph{(b)}
$ $\newline

$a_n=(\sqrt[n]{n}-1)^n$

dann $\sqrt[n]{|a_n|}=\sqrt[n]{(\sqrt[n]{n}-1)^n}=\sqrt[n]{n}-1\overset{n\rightarrow\infty}{\longrightarrow}0$

z.B. Sei $\theta_i=\frac{1}{2}=\epsilon$ $\exists N\in\mathbb{N}$

$\forall n\in\mathbb{N}_{>N}$: $|\sqrt[n]{n}-1-0|<\epsilon=\frac{1}{2}$ $\Rightarrow$ $\sqrt[n]{n}-1<\frac{1}{2}$

\paragraph{(c)}
$ $\newline

$a_n=(-1)^n\frac{n+1}{n}$

$a_n$ kein Nullfolge

nicht konv.

\paragraph{(d)}
$ $\newline

$a_N=\frac{n+4}{n^3-3n+1}$

mit Harmonische Reihe ($n>1$ konv.)

\begin{align*}
&=\frac{n}{n^3-3n+1}+\frac{4}{n^3-3n+1}\\
&\leq\frac{n}{n^3-3n}+\frac{4}{n^3-3n}\\
&=\frac{1}{n^2-3}+\frac{4}{n^3-3n}\\
&\leq\frac{1}{(n-3)^2}+\frac{1}{(n-4)^3}\hspace{1em}for\hspace{1em}n>20
\end{align*}

\newpage

\paragraph{(e)}
\begin{align*}
a_n
&=\ldots\\
&=\frac{2\sqrt{n}}{n-1}\\
&\geq\frac{2\sqrt{n}}{n}=\frac{2}{\sqrt{n}}\ div.
\end{align*}

\paragraph{(f)}
$ $\newline

\newpage

\subsection{Aufgabe 8.5(H)}

\paragraph{(a)}
$ $\newline

For
\begin{equation*}
\sum_{n=1}^{\infty}a_n=\sum_{n=1}^{\infty}\frac{n^2}{2^n}
\end{equation*}

We determine with ratio test:
\begin{align*}
\frac{a_{n+1}}{a_n}
&=\frac{(n+1)^2}{2^{n+1}}\frac{2^n}{n^2}\\
&=\frac{n^2+2n+1}{2n^2}\\
&=\frac{1}{2}+\frac{1}{n}+\frac{1}{2n^2}\\
&\mbox{for }n\rightarrow\infty\mbox{ we have}\\
\lim_{n\rightarrow\infty}\Bigg|\frac{a_{n+1}}{a_n}\Bigg|&=\frac{1}{2}<1\\
&\Rightarrow\mbox{this series is convergent}
\end{align*}

\paragraph{(b)}
$ $\newline

For
\begin{equation*}
\sum_{n=1}^{\infty}a_n=\sum_{n=1}^{\infty}\Big(1+\frac{1}{n^2}\Big)^n
\end{equation*}

We determine with ratio test:
\begin{align*}
\frac{a_{n+1}}{a_n}
&=\frac{(1+\frac{1}{(n+1)^2})^{n+1}}{(1+\frac{1}{n^2})^n}\\
&=\frac{\Big((1+\frac{1}{(n+1)^2})^{(n+1)^2}\Big)^{\frac{1}{n+1}}}{\Big((1+\frac{1}{n^2})^{n^2}\Big)^\frac{1}{n}}\\
(\mbox{substitute }\tilde{n}=n^2,\tilde{n}'=(n+1)^2)&=
\frac{\Big((1+\frac{1}{\tilde{n}'})^{\tilde{n}'}\Big)^{\frac{1}{n+1}}}{\Big((1+\frac{1}{\tilde{n}})^{\tilde{n}}\Big)^\frac{1}{n}}
\end{align*}

with $n\rightarrow\infty$, we have
\begin{equation*}
\lim_{n\rightarrow\infty}\tilde{n}=\lim_{n\rightarrow\infty}\tilde{n}'=\lim_{n\rightarrow\infty}n=\infty,\hspace{1em}\lim_{n\rightarrow\infty}\frac{1}{n+1}=\lim_{n\rightarrow\infty}\frac{1}{n}=0
\end{equation*}

therefore
\begin{equation*}
\lim_{n\rightarrow\infty}\Bigg|\frac{a_{n+1}}{a_n}\Bigg|
\lim_{n\rightarrow\infty}\Bigg|\frac{\Big((1+\frac{1}{\tilde{n}'})^{\tilde{n}'}\Big)^{\frac{1}{n+1}}}{\Big((1+\frac{1}{\tilde{n}})^{\tilde{n}}\Big)^\frac{1}{n}}\Bigg|=\lim_{n\rightarrow\infty}\Bigg|\frac{e^{\frac{1}{n+1}}}{e^\frac{1}{n}}\Bigg|=1
\end{equation*}

with $(1+\frac{1}{n^2})^n>1$, the series is divergent

\newpage

\paragraph{(c)}
$ $\newline

For
\begin{equation*}
\sum_{n=1}^{\infty}a_n=\sum_{n=1}^{\infty}\frac{1}{7^n}
\begin{pmatrix}
3n \\
n
\end{pmatrix}
\end{equation*}

We determine with ratio test
\begin{align*}
\Bigg|\frac{a_{n+1}}{a_n}\Bigg|
&=\frac{\frac{(3(n+1))!}{7^{n+1}(2(n+1))!(n+1)!}}{\frac{3n!}{7^n(2n)!n!}}\\
&=\frac{(3n+3)!(2n)!n!}{7(2n+2)!(n+1)!3n!}\\
&=\frac{\Big(\prod_{k=1}^{3n+3}k\Big)\Big(\prod_{k=1}^{2n}k\Big)\Big(\prod_{k=1}^{n}k\Big)}{7\Big(\prod_{k=1}^{2n+2}k\Big)\Big(\prod_{k=1}^{n+1}k\Big)\Big(\prod_{k=1}^{3n}k\Big)}\\
&=\frac{(3n+1)(3n+2)(3n+3)}{7(2n+1)(2n+2)(n+1)}\\
&\Rightarrow\lim_{n\rightarrow\infty}\frac{(3n+1)(3n+2)(3n+3)}{7(2n+1)(2n+2)(n+1)}=\frac{27}{28}<1\\
&\Rightarrow\mbox{this series is convergent}
\end{align*}

\paragraph{(d)}
$ $\newline

For
\begin{equation*}
\sum_{n=1}^{\infty}(-1)^na_n=\sum_{n=1}^{\infty}(-1)^n\frac{1}{\sqrt{n}}
\end{equation*}

We determine with Leibniz criterion:

\begin{align*}
|a_{n+1}|=\Bigg|\frac{1}{\sqrt{n+1}}\Bigg|&<\Bigg|\frac{1}{\sqrt{n}}\Bigg|=|a_n|\\
\Rightarrow&a_n\mbox{ monotonically decreases}
\end{align*}
\begin{align*}
\lim_{n\rightarrow\infty}a_n=0
\end{align*}

$\Rightarrow$ this series is convergent

\newpage

\subsection{Aufgabe 8.6(H)}

\begin{proof}
$ $\newline

Sei $b_n$ beschränkt, dann $\exists N\in\mathbb{N}$, $\forall n\in\mathbb{N}$ $|b_n|\leq |b_N|$\\

Sei $\lim_{n\rightarrow\infty}a_n=0$, dann $\forall\varepsilon\in\mathbb{R}$ $\exists N'\in\mathbb{N}$, $\forall n\in\mathbb{N}_{>N'}$ $a_n<\varepsilon$\\

dann $\forall b_N\in\mathbb{R}$, $\lim_{n\rightarrow\infty}a_n\cdot b_N=0$\\

dann mit Majorantenkriterium $a_n\cdot b_n\rightarrow0$

\end{proof}

\newpage

\subsection{Tutorium}

\begin{definition}[Cauchy-Folge]
$(a_n)_{n\in\mathbb{N}}$ ist CF, wenn

$\forall\epsilon>0$ $\exists N\in\mathbb{N}$, sodass $\forall m,n\in\mathbb{N}_{>N}$: $|a_n-a_m|<\epsilon$
\end{definition}

\begin{definition}[Reihe]
$\sum_{n=1}^{\infty}=(b_n)_{n\in\mathbb{N}}$, [$\forall n\in\mathbb{N}$: $b_n=\sum_{k=1}^n a_k$]

(see: partial summation convergent)
\end{definition}

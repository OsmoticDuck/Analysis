\section{19. Übungsblatt}

\paragraph{Aufgabe 19.1}

\begin{enumerate}

\item[(a)]

aus Grafik ermitteln

\begin{align*}
\int_0^2\varphi_1dx=6
\end{align*}

\item[(b)]

\begin{align*}
\int_0^2\varphi_2dx=0\frac{1}{2}+1\frac{1}{2}+\cdots=3
\end{align*}

\item[(c)]

\begin{align*}
\int_0^2\varphi_3dx=7\int_0^2\varphi_1dx-5\int_0^2\varphi_2dx=-8
\end{align*}

\item[(d)]

Es ist kein TF.

\end{enumerate}

\newpage

\paragraph{Aufgabe 19.2}

\begin{enumerate}

\item[]

zu beweisen:
\begin{align*}
0\leq\int{}^*f+\int{}^*g-\int{}^*(f+g)
\end{align*}

$\Rightarrow$
\begin{align*}
0\leq & \inf\{\int_a^b\varphi_1dx:\varphi_1\in T[a,b],\varphi\geq f\}+\inf\cdots \\
&=
\inf\{\int_a^b(\varphi_1+\varphi_2)dx:\varphi_{1,2}\in T[a,b],\varphi_1\geq f,\varphi_2\geq g\}-\cdots	
\end{align*}

\end{enumerate}

\paragraph{Aufgabe 19.3}

\begin{enumerate}

\item[]

\end{enumerate}

\newpage

\paragraph{Aufgabe 19.5(H)}

\begin{enumerate}

\item[]

Gegeben sei die Funktion
\begin{align*}
f:\mathbb{R}\rightarrow\mathbb{R},
\hspace{1em}
x\mapsto
\left\{
\begin{array}{lcl}
\sin(x) & \mbox{falls }x\geq\frac{\pi}{2} \\
a(x+b)^2+c & \mbox{falls }x<\frac{\pi}{2}
\end{array}
\right.
\end{align*}

nun offenbar ist $\sin(x)$ in $(-\infty,\frac{\pi}{2})$ differenzierbar. Die Voraussetzungen für $f$ differenzierbar sind:
\begin{enumerate}
\item[(i)]$f$ am Punkt $\frac{\pi}{2}$ stetig und differenzierbar.
\item[(ii)]$a(x+b)^2+c$ in $(\frac{\pi}{2},\infty)$ stetig und differenzierbar.
\end{enumerate}

dazu
\item[(i)]
\begin{align*}
\sin(\frac{\pi}{2})&=a(\frac{\pi}{2}+b)^2+c \\
1&=a(\frac{\pi}{2}+b)^2+c
\end{align*}
und die Ableitungen am Punkt $\frac{\pi}{2}$
\begin{align*}
\cos(\frac{\pi}{2})&=2a(\frac{\pi}{2}+b) \\
0&=2a(\frac{\pi}{2}+b)
\end{align*}

führt zur Lösung
\begin{align*}
\left\{
\begin{array}{lcl}
a=0 \\
b\in\mathbb{R} \\
c=1
\end{array}
\right.
\hspace{1em}\mbox{oder}\hspace{1em}
\left\{
\begin{array}{lcl}
a\in\mathbb{R} \\
b=0 \\
c=1
\end{array}
\right.
\end{align*}

\item[(ii)]
Nach (i) ist $f$ in $(\frac{\pi}{2},\infty)$ Konstant ($f=1$), offenbar stetig und differenzierbar.

\begin{center}


\tikzset{every picture/.style={line width=0.75pt}} %set default line width to 0.75pt        

\begin{tikzpicture}[x=0.75pt,y=0.75pt,yscale=-1,xscale=1]
%uncomment if require: \path (0,309); %set diagram left start at 0, and has height of 309

%Shape: Axis 2D [id:dp8104424795739424] 
\draw  (-2.33,177) -- (347.67,177)(103.04,57) -- (103.04,291) (340.67,172) -- (347.67,177) -- (340.67,182) (98.04,64) -- (103.04,57) -- (108.04,64)  ;
%Shape: Wave [id:dp08206399217933691] 
\draw   (2.67,177) .. controls (18.97,204.66) and (34.57,231) .. (52.67,231) .. controls (70.76,231) and (86.36,204.66) .. (102.67,177) .. controls (118.97,149.34) and (134.57,123) .. (152.67,123) .. controls (153.34,123) and (154,123.04) .. (154.67,123.11) ;
%Straight Lines [id:da6140218308409282] 
\draw    (154,123) -- (284.67,123) ;
%Straight Lines [id:da31041816577154635] 
\draw  [dash pattern={on 0.84pt off 2.51pt}]  (154,123) -- (154,176) ;
%Straight Lines [id:da886229789117055] 
\draw  [dash pattern={on 0.84pt off 2.51pt}]  (102.67,123) -- (154,123) ;

% Text Node
\draw (156,179) node [anchor=north west][inner sep=0.75pt]   [align=left] {$\frac{\pi}{2}$};
% Text Node
\draw (339,186) node [anchor=north west][inner sep=0.75pt]   [align=left] {$x$};
% Text Node
\draw (86,54) node [anchor=north west][inner sep=0.75pt]   [align=left] {$y$};
% Text Node
\draw (88,116) node [anchor=north west][inner sep=0.75pt]   [align=left] {$1$};


\end{tikzpicture}
\end{center}

\end{enumerate}

\newpage

\paragraph{Aufgabe 19.6(H)}

\begin{enumerate}

\item[(a)]

Legendreschen Polynome
\begin{align*}
L_n:\mathbb{R}\rightarrow\mathbb{R},
\hspace{1em}
x\mapsto\frac{1}{2^n n!}\cdot\frac{d^n}{dx^n}[(x^2-1)^n]
\end{align*}

$L_0,L_1,L_2$ sind
\begin{align*}
L_0:\hspace{1em}
x\mapsto &\frac{1}{2^0 0!}\cdot\frac{d^0}{dx^0}[(x^2-1)^0] \\
=&1 \\
L_1:\hspace{1em}
x\mapsto &\frac{1}{2^1 1!}\cdot\frac{d^1}{dx^1}[(x^2-1)^1] \\
=&x \\
L_2:\hspace{1em}
x\mapsto &\frac{1}{2^2 2!}\cdot\frac{d^2}{dx^2}[(x^2-1)^2] \\
=&\frac{1}{2}(3x^2-1)
\end{align*}

\item[(b)]

Legendreschen Polynome
\begin{align*}
L_n:\mathbb{R}\rightarrow\mathbb{R},
\hspace{1em}
x\mapsto &\frac{1}{2^n n!}\cdot\frac{d^n}{dx^n}[(x^2-1)^n] \\
=&\frac{1}{2^n n!}\cdot\frac{d^n}{dx^n}[(x+1)^n(x-1)^n]
\end{align*}

sei $f(x)=(x+1)^n(x-1)^n$, wir wenden Leibniz-Regel an:
\begin{align*}
f(x)=&(x+1)^n(x-1)^n \\
f^1(x)=&\frac{d}{dx}[(x+1)^n](x-1)^n+(x+1)^n\frac{d}{dx}[(x-1)^n] \\
f^2(x)=&\frac{d^2}{dx^2}[(x+1)^n](x-1)^n+\frac{d}{dx}[(x+1)^n]\frac{d}{dx}[(x-1)^n]+ \\
& (x+1)^n\frac{d^2}{dx^2}[(x-1)^n]+\frac{d}{dx}[(x+1)^n]\frac{d}{dx}[(x-1)^n] \\
\vdots & \\
\mbox{(mit Binomial-Koeff.)}\hspace{1em}
f^n(x)=&
\begin{pmatrix}
n \\
0
\end{pmatrix}
[(x+1)^n]\frac{d^n}{dx^n}[(x-1)^n]+
\begin{pmatrix}
n \\
1
\end{pmatrix}
\frac{d}{dx}[(x+1)^n]\frac{d^{n-1}}{dx^{n-1}}[(x-1)^n]+ \\
&
\begin{pmatrix}
n \\
2
\end{pmatrix}
\frac{d^2}{dx^2}[(x+1)^n]\frac{d^{n-2}}{dx^{n-2}}[(x-1)^n]+\cdots+ \\
&
\begin{pmatrix}
n \\
n-1
\end{pmatrix}
\frac{d^{n-1}}{dx^{n-1}}[(x+1)^n]\frac{d}{dx}[(x-1)^n]+
\begin{pmatrix}
n \\
n
\end{pmatrix}
\frac{d^n}{dx^n}[(x+1)^n](x-1)^n
\end{align*}

dann gilt
\begin{align*}
L_n=\frac{1}{2^n n!}
\bigg[
\sum_{i=0}^n
\begin{pmatrix}
n \\
i
\end{pmatrix}
\frac{d^i}{dx^i}[(x+1)^n]\frac{d^{n-i}}{dx^{n-i}}[(x-1)^n]
\bigg]
\end{align*}

\newpage

\begin{enumerate}

\item[(i)]$x=1$

wenn $i\neq 0$ ist, es gilt
\begin{align*}
\frac{d^{n-i}}{dx^{n-i}}[(x-1)^n]=0
\end{align*}

dann
\begin{align*}
L_n(1)
&=\frac{1}{2^n n!}(x+1)^n\frac{d^n}{dx^n}[(x-1)^n] \\
&=\frac{1}{n!}[n\cdot (n-1)\cdot (n-2)\cdots 2\cdot 1] \\
&=\frac{1}{n!}n!=1
\end{align*}

\item[(ii)]$x=-1$

wenn $i\neq 0$ ist, es gilt
\begin{align*}
\frac{d^{n-i}}{dx^{n-i}}[(x+1)^n]=0
\end{align*}

dann
\begin{align*}
L_n(-1)
&=\frac{1}{2^n n!}\frac{d^n}{dx^n}[(x+1)^n](x-1)^n \\
&=\frac{(-1)^n}{n!}[n\cdot (n-1)\cdot (n-2)\cdots 2\cdot 1] \\
&=\frac{(-1)^n}{n!}n!=(-1)^n
\end{align*}

\end{enumerate}

\item[(c)]

Sei $f(x):=(x+1)^n(x-1)^n=[(x+1)(x-1)]^n=(x^2-1)^n$

Es gilt $f(1)=f(-1)=0$, nach S.v.Rolle hat $f'$ in $(-1,1)$ mindestens einen Nullpunkt $x_{1,1}$. Wir zerbrechen das Intervall $(-1,1)$ mit dem Punkt $x_{1,1}$, erhalten $(-1,x_{1,1})$ und $(x_{1,1},1)$, es gilt $f'(1)=f'(-1)=f'(x_{1,1})=0$ dann nach S.v.Rolle hat $f''$ in zwei Teilintervalle je einen Nullpunkt $x_{2,1}$ und $x_{2,2}$ (mindestens zwei). Dann können wir noch mal zerbrechen und drei Teilintervalle $(-1,x_{2,1})$ $(x_{2,1},x_{2,2})$ und $(x_{2,2},1)$ erhalten, nach S.v.Rolle hat $f'''$ mindestens drei Nullpunkte. Bei der $n-$te Ableitung hat $f^n$ mindestens $n$ Nullpunkten. Für ein $n$-te Polynom ist maximal $n$ Nullpunkte möglich. Induktiv können wir zusammenfassen, dass $f^n$ im $(-1,1)$ insgesamt $n$ Nullpunkte hat.

Dann hat Legendre Polynomial $L_n=\frac{1}{2^n n!}f^n(x)$ im $(-1,1)$ insgesamt $n$ Nullpunkte.

\end{enumerate}

\newpage

\paragraph{Tutorium}

\begin{enumerate}

\item[]

$f:[a,b]\rightarrow\mathbb{R},f\in R[a,b]$

Oberintegral:
\begin{align*}
\int_a^b {}^* f(x)dx:=\inf\{\int_a^b\varphi(x)dx|\varphi\in J[a,b],\varphi\geq f\}
\end{align*}

Unterintegral:
\begin{align*}
\int_a^b {}_* f(x)dx:=\sup\{\int_a^b\varphi(x)dx|\varphi\in J[a,b],\varphi\geq f\}
\end{align*}

dann
\begin{align*}
Unterint.<Oberint.
\end{align*}

\begin{align*}
f\in R[a,b]\Rightarrow Oberint.=Unterint.=\int_a^b f(x)dx
\end{align*}

\begin{align*}
J[a,b]\subseteq R[a,b]
\end{align*}

\begin{align*}
f:[a,b]\rightarrow\mathbb{R}\ stetig\ monoton
\end{align*}

sei $c\in[a,b]$
\begin{align*}
\int_a^b=\int_a^c+\int_c^b
\end{align*}

\end{enumerate}
\section{2. Übungsblatt: $Mengen$, $vollständige\ Induktion$}

\subsection{Aufgabe 2.1}

\paragraph{(a)}

\subparagraph{(i)}
\begin{align*}
A\cap B&=\{2\}\\
D\setminus B&=\{5\}\\
A\cup C&=\{1,2,3,4\}\\
A\cap D&=\emptyset\\
C\setminus D&=\{1,2\}\\
D\setminus C&=\{5,6\}\\
A\setminus(B\cap C)&=\{1,3\}\\
C\setminus (B\setminus A)&=\{1,2\}\\
(C\setminus B)\setminus A&=\emptyset
\end{align*}

\subparagraph{(ii)}
\begin{align*}
\mathcal{P}(A)&=\{\emptyset,\{1\},\{2\},\{3\},\{1,2\},\{1,3\},\{2,3\},A\}\\
\mathcal{P}(B)\setminus\{B\}&=\{\emptyset,\{2\},\{4\},\{6\},\{2,4\},\{2,6\},\{4,6\}\}\\
\mathcal{P}(C)\cap\mathcal{P}(D)&=\{\emptyset,\{4\}\}
\end{align*}

\subparagraph{(iii)}
\begin{align*}
A\times B&=\{(1,2),(1,4),(1,6),(2,2),(2,4),(2,6),(3,2),(3,4),(3,6)\}\\
(C\times D)\setminus(A\times B)&=\{(1,5),(2,5),(4,4),(4,5),(4,6)\}\\
(A\times B)\cap(C\times D)&=\{(1,4),(1,6),(2,4),(2,6)\}
\end{align*}

\paragraph{(b)}
$ $\newline
$M$ enthält 5 Elementen.
\begin{remark}
$\{\emptyset,\emptyset\}$ ist $\{\emptyset\}$, deshalb verschwindet.
\end{remark}
\begin{align*}
\emptyset\in M&,\ \emptyset\subseteq M\\
\{\emptyset\}\in M&,\ \{\emptyset\}\subseteq M\\
\{\{\emptyset\}\}\in M&,\ \{\{\emptyset\}\}\subseteq M\\
\{\emptyset,\{\emptyset\}\}\in M&,\ \{\emptyset,\{\emptyset\}\}\subseteq M\\
\{\{\{\emptyset\}\}\}\notin M&,\ \{\{\{\emptyset\}\}\}\subseteq M\\
\{\{\emptyset\},\{\{\emptyset\}\}\}\notin M&,\ \{\{\emptyset\},\{\{\emptyset\}\}\}\subseteq M
\end{align*}

\newpage

\subsection{Aufgabe 2.2}

\paragraph{(a)}
\begin{proof}
$ $\newline
\begin{align}
B\cap C&=\{x\in B\wedge x\in C\}\\
\Rightarrow A\cup(B\cap C)&=\{x\in A\vee(x\in B\wedge x\in C)\}
\end{align}
\begin{align}
A\cup B&=\{x\in A\vee x\in B\}\\
A\cup C&=\{x\in A\vee x\in C\}\\
\Rightarrow (A\cup B)\cap(A\cup C)&=\{(x\in A\vee x\in B)\wedge(x\in A\vee x\in C)\}\\
&=\{x\in A\vee(x\in B\wedge x\in C)\}=A\cup(B\cap C)
\end{align}
\begin{align}
\Rightarrow A\cup(B\cap C)&=(A\cup B)\cap(A\cup C)
\end{align}
\end{proof}

\paragraph{(b)}
\begin{proof}
$ $\newline
\begin{align}
B\cap C&=\{x\in B\wedge x\in C\}\\
\Rightarrow A\setminus(B\cap C)&=\{x\in A\wedge\neg(x\in B\wedge x\in C)\}
\end{align}
\begin{align}
A\setminus B&=\{x\in A\wedge(x\notin B)\}\\
A\setminus C&=\{x\in A\wedge(x\notin C)\}\\
\Rightarrow (A\setminus B)\cup(A\setminus C)&=\{(x\in A\wedge(x\notin B))\vee(x\in A\wedge(x\notin C))\}\\
&=\{x\in A\wedge((x\notin B)\vee(x\notin C))\}\\
&=\{x\in A\wedge\neg(x\in B\wedge x\in C)\}=A\setminus(B\cap C)
\end{align}
\begin{align}
\Rightarrow A\setminus(B\cap C)&=(A\setminus B)\cup(A\setminus C)
\end{align}
\end{proof}

\newpage

\subsection{Aufgabe 2.3}

\paragraph{(a)}
\begin{remark}
sehe Tutorium
\end{remark}
\begin{proof}
$ $\newline
\begin{align}
A\setminus B&=\{x\in A\wedge(x\notin B)\}\\
\Rightarrow (A\setminus B)\setminus C&=\{(x\in A\wedge(x\notin B))\wedge(x\notin C)\}\\
&=\{x\in A\wedge(x\notin B\wedge x\notin C)\}
\end{align}
\begin{align}
B\setminus C&=\{x\in B\wedge(x\notin C)\}\\
\Rightarrow A\setminus(B\setminus C)&=\{x\in A\wedge\neg(x\in B\wedge(x\notin C))\}\\
&=\{x\in A\wedge(x\notin B\vee x\in C)\}
\end{align}
dann:
\begin{align}
&x\in A\wedge x\notin B\wedge x\notin C\\
&\Rightarrow x\in A\wedge(x\notin B\vee x\in C)\Rightarrow x\in A\setminus(B\setminus C)
\end{align}
\begin{align}
\Rightarrow (A\setminus B)\setminus C\subseteq A\setminus(B\setminus C)
\end{align}
\end{proof}

\paragraph{(b)}

\subparagraph{(i)}
\begin{proof}
$ $\newline
\begin{align}
\mathcal{P}(A)\cap\mathcal{P}(B)&=\{(S\in\mathcal{P}(A))\wedge(S\in\mathcal{P}(B)\}\\
&=\{(S\subseteq A)\wedge(S\subseteq B)\}\\
&=\{((x\in S)\wedge(x\in A))\wedge((x\in S)\wedge(x\in B))\}\\
&=\{(\forall x\in S)\wedge(x\in A)\wedge(x\in B)\}
\end{align}
\begin{align}
\mathcal{P}(A\cap B)&=\{S\in\mathcal{P}(A\cap B)\}\\
&=\{(\forall x\in S)\wedge(x\in A)\wedge(x\in B)\}\\
&=\mathcal{P}(A)\cap\mathcal{P}(B)
\end{align}
\begin{align}
\Rightarrow \mathcal{P}(A)\cap\mathcal{P}(B)\subseteq\mathcal{P}(A\cap B)
\end{align}
\end{proof}

\newpage

\subparagraph{(ii)}
\begin{proof}
$ $\newline
\begin{align}
\mathcal{P}(A)\cup\mathcal{P}(B)&=\{(S\in\mathcal{P}(A))\vee(S\in\mathcal{P}(B))\}\\
&=\{((\forall x\in S)\wedge(x\in A))\vee((\forall x\in S)\wedge(x\in B))\}\\
&=\{\forall x\in S\wedge(x\in A\vee x\in B)\}
\end{align}
\begin{align}
\mathcal{P}(A\cup B)&=
\end{align}
\end{proof}

\newpage

\subsection{Aufgabe 2.4}

\paragraph{(a)}
\begin{proof}
$ $\newline
Wir beweisen nun $\forall(n\geq 5)\in\mathbb{N}$, $2^n>n^2$.\\
$ $\newline
$Induktionsanfang:$\\
Sei $n=5$, dann gilt:
\begin{equation*}
2^5=32>25=5^2
\end{equation*}
$ $\newline
$Induktionsvorraussetzung:$\\
$\forall(n\geq 5)\in\mathbb{N}$, $2^n>n^2$\\
$ $\newline
$Induktionsschritt:$\\
Wir setzen $\tilde{n}=n+1$, dann gilt:
\begin{equation*}
2^{\tilde{n}}=2^{n+1}=2^n\cdot 2\overset{\mathbf{IV}}{>} n^2\cdot 2=n^2+n^2=n^2+2n+1+n^2-2n-1=(n+1)^2+(n-1)^2-2
\end{equation*}
Offenb. $(n-1)^2-2>0$ für $\forall(n\geq 5)\in\mathbb{N}$, dann gilt:
\begin{equation*}
2^{n+1}>(n+1)^2
\end{equation*}
%Hier müssen wir beweisen, dass $\forall(n\geq 5)\in\mathbb{N}$, $n^2>2n+1$.
%\begin{proof}
%$ $\newline
%Wir beweisen nun $\forall(n\geq 5)\in\mathbb{N}$, $n^2>2n+1$.
%$ $\newline
%$Induktionsanfang:$\\
%Sei $n=5$, dann gilt:
%\begin{equation*}
%5^2=25>11=2\cdot 5+1
%\end{equation*}
%$ $\newline
%$Induktionsvorrausetzung:$\\
%$\forall(n\geq 5)\in\mathbb{N}$, $n^2>2n+1$\\
%$ $\newline
%$Induktionsschritt:$\\
%Wir setzen $\tilde{n}=n+1$, dann gilt:
%\begin{equation*}
%(n+1)^2=n^2+2n+1>4n+2>2n+3
%\end{equation*}
%\end{proof}
\end{proof}

\paragraph{(b)}
\begin{proof}
$ $\newline
Wir beweisen nun $\forall n\in\mathbb{N}$, gilt:
\begin{equation*}
\sum_{k=0}^{n}k^2=\frac{n(n+1)(2n+1)}{6}
\end{equation*}
$ $\newline
$Induktionsanfang:$\\
Sei $n=0$, dann gilt:
\begin{equation*}
\sum_{k=0}^{0}k^2=0^2=0=\frac{0(0+1)(2\cdot 0+1)}{6}
\end{equation*}
$ $\newline
$Induktionsvorraussetzung:$\\
Sei $\forall n\in\mathbb{N}$, gilt:
\begin{equation*}
\sum_{k=0}^{n}k^2=\frac{n(n+1)(2n+1)}{6}
\end{equation*}

\newpage

$Induktionsschritt:$\\
Wir setzen $\tilde{n}=n+1$, dann gilt:
\begin{align*}
\sum_{k=0}^{\tilde{n}}k^2=\sum_{k=0}^{n+1}k^2&=0^2+1^2+\cdots +n^2+(n+1)^2\\
&=\sum_{k=0}^{\tilde{n}}k^2+(n+1)^2\\
&\overset{\mathbf{IV}}{=}\frac{n(n+1)(2n+1)}{6}+(n+1)^2\\
&=\frac{n(n+1)(2n+1)+6(n+1)^2}{6}\\
&=\frac{(n+1)(n(2n+1)+6(n+1))}{6}\\
&=\frac{(n+1)((n+1)+1)(2(n+1)+1)}{6}
\end{align*}
\end{proof}

\newpage

\subsection{Aufgabe 2.5(H)}

\paragraph{(a)}

\subparagraph{(i)}
$\forall m\in M$, $\exists f\in F$, die mit $m$ getanzt hat.

\subparagraph{(ii)}
$\exists f\in F$, die mit $\forall m\in M$ getanzt hat.

\paragraph{(b)}
$ $\newline
When (ii) is true, (i) must be true.\\
When (ii) is false, (i) can be either true or false.\\
This corresponds $logical\ implication$\\
$conclusion:$ (ii)$\Rightarrow$(i)

\paragraph{(c)}

\subparagraph{(i)}
$\exists m\in M$, $\nexists f\in F$, die mit $m$ getanzt hat.

\subparagraph{(ii)}
$\forall f\in F$, $\exists m\in M$, die nicht mit $f$ getanzt hat.

\newpage

\subsection{Aufgabe 2.6(H)}

\paragraph{(a)}
\begin{proof}
$ $\newline
\begin{align}
A\bigtriangleup B:&=(A\setminus B)\cup(B\setminus A)\\
A\setminus B&=\{x\in A\wedge(x\notin B)\}\\
B\setminus A&=\{x\in B\wedge(x\notin A)\}\\
\Rightarrow A\bigtriangleup B&=\{(x\in A\wedge(x\notin B))\vee(x\in B\wedge(x\notin A))\}
\end{align}
\begin{align}
A\cup B&=\{x\in A\vee x\in B\}\\
A\cap B&=\{x\in A\wedge x\in B\}\\
\Rightarrow (A\cup B)\setminus(A\cap B)&=\{(x\in A\vee x\in B)\wedge\neg(x\in A\wedge x\in B)\}\\
&=\{(x\in A\wedge(x\notin B))\vee(x\in B\wedge(x\notin A))\}=A\bigtriangleup B
\end{align}
\begin{align}
\Rightarrow A\bigtriangleup B&=(A\cup B)\setminus(A\cap B)
\end{align}
\end{proof}

\newpage

\paragraph{(b)}
\begin{proof}
$ $\newline
\begin{align}
A\bigtriangleup B:&=(A\setminus B)\cup(B\setminus A)\\
\Rightarrow (A\bigtriangleup B)\bigtriangleup C&=(((A\setminus B)\cup(B\setminus A))\setminus C)\cup(C\setminus((A\setminus B)\cup(B\setminus A)))\\
und\ A\bigtriangleup(B\bigtriangleup C)&=(A\setminus((B\setminus C)\cup(C\setminus B)))\cup(((B\setminus C)\cup(C\setminus B))\setminus A)
\end{align}
\begin{align}
A\setminus B=&\{x\in A\wedge(x\notin B)\}\\
B\setminus A=&\{x\in B\wedge(x\notin A)\}\\
\Rightarrow A\bigtriangleup B=((A\setminus B)\cup(B\setminus A))=&\{(x\in A\wedge(x\notin B))\vee(x\in B\wedge(x\notin A))\}
\end{align}
dann:
\begin{align}
\begin{split}
\Rightarrow (A\bigtriangleup B)\bigtriangleup C=&\{(((x\in A\wedge(x\notin B))\vee(x\in B\wedge(x\notin A)))\wedge(x\notin C))\\
&\vee(x\in C\wedge\neg((x\in A\wedge(x\notin B))\vee(x\in B\wedge(x\notin A))))\}
\end{split}\\
\begin{split}
=&\{(((x\in A\vee x\in B)\wedge\neg(x\in A\wedge x\in B))\wedge(x\notin C))\\
&\vee(x\in C\wedge\neg((x\in A\vee x\in B)\wedge\neg(x\in A\wedge x\in B)))\}
\end{split}\\
\begin{split}
=&\{(x\in A\vee x\in B\vee x\in C)\\
&\wedge\neg(((x\in A\wedge x\in B)\wedge(x\notin C))\\
&\wedge(((x\in B\wedge x\in C)\wedge(x\notin A))\\
&\wedge(((x\in C\wedge x\in A)\wedge(x\notin B)))\}
\end{split}\\
=&(A\cup B\cup C)\setminus(((A\cap B)\cup(B\cap C)\cup(C\cap A))\setminus(A\cap B\cap C))
\end{align}
analog:
\begin{align}
\begin{split}
\Rightarrow A\bigtriangleup(B\bigtriangleup C)=&\{(x\in A\wedge\neg((x\in B\wedge(x\notin C))\vee(x\in C\wedge(x\notin B))))\\
&\vee(((x\in B\wedge(x\notin C))\vee(x\in C\wedge(x\notin B)))\wedge(x\notin A))\}
\end{split}\\
\begin{split}
=&\{(x\in A\wedge\neg((x\in B\vee x\in C)\wedge\neg(x\in B\wedge x\in C)))\\
&\vee(((x\in B\vee x\in C)\wedge\neg(x\in B\wedge x\in C))\wedge(x\notin A))\}
\end{split}\\
\begin{split}
=&\{(x\in A\vee x\in B\vee x\in C)\\
&\wedge\neg(((x\in A\wedge x\in B)\wedge(x\notin C))\\
&\wedge(((x\in B\wedge x\in C)\wedge(x\notin A))\\
&\wedge(((x\in C\wedge x\in A)\wedge(x\notin B)))\}
\end{split}\\
=&(A\cup B\cup C)\setminus(((A\cap B)\cup(B\cap C)\cup(C\cap A))\setminus(A\cap B\cap C))
\end{align}
\begin{align}
\Rightarrow (A\bigtriangleup B)\bigtriangleup C&=A\bigtriangleup(B\bigtriangleup C)
\end{align}
\end{proof}

\newpage

\subsection{Tutorium}
$A\subseteq B$ $\Leftrightarrow$ $\forall x:(x\in A\Rightarrow x\in B)$\\
$A\setminus B$ $\Rightarrow$ $\forall x:(x\in A\wedge x\notin B)$\\
$A\setminus B=\{x\in M\ |\ x\in A\wedge x\notin B\}$

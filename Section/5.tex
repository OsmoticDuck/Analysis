\section{5. Übungsblatt:}

\subsection{Aufgabe 5.1}

$n^{n+1}
\begin{pmatrix}
= \\
< \\
>
\end{pmatrix}
(n+1)^n$\\

$n=3$, $LS=3^4=81$ $RS=4^3=64$\\

Vermutung: $n^{n+1}>(n+1)^n$\\

\begin{equation*}
\Leftrightarrow1>\frac{(n+1)^n}{n^{n+1}}=\frac{1}{n}(1+\frac{1}{n})^n
\end{equation*}
\begin{align*}
&=\frac{1}{n}\Bigg(\sum_{k=0}^n
\begin{pmatrix}
n \\
k
\end{pmatrix}
1^{n-k}
(\frac{1}{n})^k
\Bigg)\\
&=\frac{1}{n}\Bigg(\sum_{k=0}^n\frac{n!}{(n-k)!k!}\frac{1}{n^k}\Bigg)\\
(n-k)!\mbox{ hat }k\mbox{ Faktoren}&=\frac{1}{n}\Bigg(\sum_{k=0}^n
\underset{\leq 1}{\underbrace{\prod_{i=1}^k\frac{n-k+i}{n}}}\cdot\frac{1}{k!}\Bigg)\\
&\leq\frac{1}{n}\Bigg(\sum_{k=0}^n\frac{1}{k!}\Bigg)\\
&=\frac{1}{n}\Bigg(\frac{5}{2}+\underset{\leq\frac{n-2}{6}}{\underbrace{\sum_{k=3}^n\underset{\leq\frac{1}{6}}{\underbrace{\frac{1}{k!}}}}}\Bigg)\\
&\leq\frac{1}{6}+\frac{13}{6n}\\
n\in\mathbb{N}_{>3}&=\frac{1}{6}+\frac{13}{18}=\frac{16}{18}<1
\end{align*}

\newpage

\subsection{Aufgabe 5.2}

$\frac{a}{b}=a\cdot b^{-1}$ mit $b\neq a$

\paragraph{(a)}

\subparagraph{(i)}
\begin{proof}
$ $\newline

z.z. $\frac{a}{b}\cdot\frac{c}{d}=\frac{ac}{bd}$\\
\begin{align*}
&(a\cdot b^{-1})\cdot(c\cdot d^{-1})\\
&a\cdot(b^{-1}(c\cdot d^{-1}))\\
&a(b^{-1}(d^{-1}\cdot c))\\
&a((b^{-1}\cdot d^{-1})\cdot c)\\
&a(c\cdot(b^{-1}\cdot d^{-1}))\\
&(a\cdot c)\cdot(b^{-1}\cdot d^{-1})\\
&(a\cdot c)\cdot(b\cdot d)^{-1}
\end{align*}
\end{proof}

\subparagraph{(ii)}
\begin{proof}
$ $\newline

$\frac{\frac{a}{b}}{\frac{c}{d}}$\\

\begin{align*}
LS=\frac{(a\cdot b^{-1})}{c\cdot d^{-1}}&=(a\cdot b^{-1})((c\cdot d^{-1})^{-1})\\
&=(a\cdot b^{-1})(c^{-1}\cdot d)\\
&=a\cdot(b^{-1}\cdot c^{-1})\cdot d\\
&=a\cdot d\cdot(b^{-1}\cdot c^{-1})\\
&=\ldots
\end{align*}
\end{proof}

\paragraph{(b)}
$ $\newline

\href{https://en.wikipedia.org/wiki/Field_(mathematics)#Classic_definition}{Field axioms}\\

\url{https://en.wikipedia.org/wiki/Field_(mathematics)#Classic_definition}

\newpage

\paragraph{(c)}

\newpage

\subsection{Aufgabe 5.3}

\newpage

\subsection{Aufgabe 5.4}

\paragraph{(a)}
$|x-a|<\epsilon$ $\Leftrightarrow$ $x\in(a-\epsilon,a+\epsilon)$\\

dann $x\geq a$ oder $x<a$

\paragraph{(b)}

\paragraph{(c)}
$\frac{1}{x}<\frac{1}{x+1}$\\

falls $x>0$, dann $x+1<x$\\

$((x+1)+x)\frac{1}{x}<(x+1)x\frac{1}{x+1}$\\

dann $x+1<x$ $contradiction$ denn $\forall x\in\mathbb{R}$, $x+1>x$

!!!!!!3 Falle.

\subsection{Aufgabe 5.5}

\newpage

\subsection{Aufgabe 5.6(H)}

\paragraph{(a)}
\begin{proof}
$ $\newline

z.z. $\frac{ad}{bd}=\frac{a}{b}$
\begin{align*}
\frac{ad}{bd}
&=(ad)(bd)^{-1}\\
&=(ad)(b^{-1}d^{-1})\\
&=a(d(b^{-1}d^{-1}))\\
&=a(d(d^{-1}b^{-1}))\\
&=a((dd^{-1})b^{-1})\\
&=ab^{-1}\\
&=\frac{a}{b}
\end{align*}
\end{proof}

\paragraph{(b)}
\begin{proof}
$ $\newline

z.z. $\frac{a}{b}+\frac{c}{d}=\frac{ad+bc}{bd}$
\begin{align*}
\frac{ad+bc}{bd}
&=(ad+bc)(bd)^{-1}\\
&=(ad)(bd)^{-1}+(bc)(bd)^{-1}\\
&=(ad)(b^{-1}d^{-1})+(bc)(b^{-1}d^{-1})\\
&=(ad)(d^{-1}b^{-1})+(cb)(b^{-1}d^{-1})\\
&=a(d(d^{-1}b^{-1}))+c(b(b^{-1}d^{-1}))\\
&=a((dd^{-1})b^{-1})+c((bb^{-1})d^{-1})\\
&=ab^{-1}+cd^{-1}\\
&=\frac{a}{b}+\frac{c}{d}
\end{align*}
\end{proof}

\newpage

\subsection{Aufgabe 5.7(H)}

$Ohne\ weitere\ Meldung\ sei\ x\in\mathbb{R}$

\paragraph{(a)}
$ $\newline

Wenn $x-3>0$, daraus folgt $x>3$ und $x+1>0$, dann
\begin{align*}
\frac{2}{x+1}&<\frac{1}{x-3}\\
2(x-3)&<x+1\\
x&<7\hspace{1em}widerspruchsfrei\\
\Rightarrow\hspace{1em}3<x&<7
\end{align*}

Wenn $x-3<0$ und $x+1>0$, daraus folgt $-1<x<3$, dann
\begin{align*}
\frac{2}{x+1}&<\frac{1}{x-3}\\
2(x-3)&>x+1\\
x&>7\hspace{1em}\lightning
\end{align*}

Wenn $x+1<0$, daraus folgt $x<-1$ und $x-3<0$, dann
\begin{align*}
\frac{2}{x+1}&<\frac{1}{x-3}\\
2(x-3)&<x+1\\
x&<7\hspace{1em}widerspruchsfrei\\
\Rightarrow\hspace{1em}x&<-1
\end{align*}

$Folgerung:$ $(x<-1)\vee(3<x<7)$

\paragraph{(b)}
$ $\newline

Wenn $x-3\geq0$, daraus folgt $x\geq3$ und $x+1>0$, dann
\begin{align*}
|x+1|+|x-3|&<6\\
x+1+x-3&<6\\
x&<4\hspace{1em}widerspruchsfrei\\
\Rightarrow\hspace{1em}3\leq x&<4
\end{align*}

Wenn $x-3<0$ und $x+1\geq0$, daraus folgt $-1\leq x<3$, dann
\begin{align*}
|x+1|+|x-3|&<6\\
x+1+3-x&<6\\
4&<6\hspace{1em}widerspruchsfrei\\
\Rightarrow\hspace{1em}-1\leq x&<3
\end{align*}

Wenn $x+1<0$, daraus folgt $x<-1$ und $x-3<0$, dann
\begin{align*}
|x+1|+|x-3|&<6\\
-x-1+3-x&<6\\
-2&<x\hspace{1em}widerspruchsfrei\\
\Rightarrow\hspace{1em}-2<x&<-1
\end{align*}

$Folgerung:$ $-2<x<4$

\newpage

\paragraph{(c)}
$ $\newline

Wenn $|x+3|>1$, $||x+3|-1|=|x+3|-1$, daraus folgt $-2\leq x$ oder $x\leq -4$\\

$-2\leq x$:

$||x+3|-1|=|x+3|-1=x+3-1=x+2=2$ folgt $x=0$ $widerspruchsfrei$\\

$x\leq -4$:

$||x+3|-1|=|x+3|-1=-x-3-1=-x-4=2$ folgt $x=-6$ $widerspruchsfrei$\\

Wenn $|x+3|<1$, $||x+3|-1|=1-|x+3|$, daraus folgt $-4<x<-2$\\

$-4<x<-3$:

$||x+3|-1|=1-|x+3|=1+x+3=x+4=2$ folgt $x=-2$ $\lightning$\\

$-3\leq x<-2$:

$||x+3|-1|=1-|x+3|=1-x-3=-x-2=2$ folgt $x=-4$ $\lightning$\\

$Folgerung:$ $(x=0)\vee(x=-6)$

\section*{Blatt 20}

\paragraph{Aufgabe 20.1}

\begin{enumerate}

\item[]

(*):
\begin{align*}
\end{align*}

Sei $\epsilon>0$ gegeb. und $N\in\mathbb{N}$ sei gemaess (*) gewahlt.

f.a. $n\geq N$ gilt
\begin{align*}
\varphi_n(x)-\epsilon'\leq f(x)\leq\varphi_n(x)+\epsilon'
\end{align*}
f.a. $x\in[a,b]$

Die folgt
\begin{align*}
\psi^\pm_n: [a,b]\rightarrow\mathbb{R}\ mit\ \psi^\pm_n(x)=\varphi_n(x)\pm\epsilon'
\end{align*}
gehoren zu $J[a,b]$

mit $\psi^-_n\leq g\leq\psi^+_n$ auf $[a,b]$

\begin{align*}
\int_a^b(\psi^+_n-\psi^-_n)(x)dx&=\int_a^b(\psi_n(x)+\epsilon')-(\psi_n(x)-\epsilon')dx \\
&=\int_a^b2\epsilon'dx=2\epsilon'(b-a)=\frac{\epsilon}{2(b-a)}2(b-a)=\epsilon
\end{align*}

fuer $n\geq N$
\begin{align*}
\int_a^b\psi^-_n(xd)dx\leq\sup\{\int_a^b\psi^-(x)dx|\psi^-\in J[a,b],\psi^+\leq f\}=\int_a^b {}_xf(x)dx\leq&\int_a^b{}^yf(x)dx=\inf\{\int_a^b\psi^+(x)|\psi^+\in J[a,b],\psi^+\geq f\} \\
\leq& \int_a^b\psi^+_n(x)=\int_a^b\psi_n(x)+\epsilon'dx=\int_a^b\psi_n(x)dx+\frac{\epsilon}{2}
\end{align*}

\end{enumerate}

\newpage

\paragraph{Aufgabe 20.5(H)}

\begin{enumerate}

\item[]

Wir ermitteln das folgende Integral mittels Riemannsche Summe
\begin{align*}
\int_a^b e^x dx
\end{align*}

Sei $n\in\mathbb{N}$, dann ist
\begin{align*}
x_k=\frac{k(b-a)}{n}+a,
\hspace{1em}
k\in\mathbb{N}_{[0,n]}
\end{align*}
eine äquidistante Unterteilung von $[a,b]$, das heißt $a=x_0<x_1<\cdots<x_n=b$

dann gilt
\begin{align*}
S_n
&=\frac{b-a}{n} \sum_{k=1}^n \exp\bigg(\frac{k(b-a)}{n}+a\bigg) \\
&=\frac{(b-a)\exp(a)}{n} \sum_{k=1}^n \bigg(\exp\bigg(\frac{b-a}{n}\bigg)\bigg)^k  \\
&=\frac{(b-a)\exp(a)}{n} \frac{\exp\bigg(\frac{b-a}{n}\bigg)(\exp(b-a)-1)}{\exp\bigg(\frac{b-a}{n}\bigg)-1} \\
&=\frac{(a-b)(e^a-e^b)}{n}+\frac{(a-b)(e^a-e^b)e^{\sfrac{a}{n}}}{n(e^{\sfrac{b}{n}}-e^{\sfrac{a}{n}})}
\end{align*}

es gilt
\begin{align*}
&\lim_{n\rightarrow\infty}
\bigg(\frac{(a-b)(e^a-e^b)}{n}+\frac{(a-b)(e^a-e^b)e^{\sfrac{a}{n}}}{n(e^{\sfrac{b}{n}}-e^{\sfrac{a}{n}})}\bigg) \\
=&
\lim_{n\rightarrow\infty}
\bigg(\frac{(a-b)(e^a-e^b)}{n}\bigg)+
\lim_{n\rightarrow\infty}
\bigg(\frac{(a-b)(e^a-e^b)e^{\sfrac{a}{n}}}{n(e^{\sfrac{b}{n}}-e^{\sfrac{a}{n}})}\bigg) \\
=&
\lim_{n\rightarrow\infty}
\bigg(\frac{(a-b)(e^a-e^b)e^{\sfrac{a}{n}}}{n(e^{\sfrac{b}{n}}-e^{\sfrac{a}{n}})}\bigg) \\
=&
(e^a-e^b) \lim_{n\rightarrow\infty}
\bigg(\frac{(a-b)e^{\sfrac{a}{n}}}{n(e^{\sfrac{b}{n}}-e^{\sfrac{a}{n}})}\bigg) \\
=&
(e^a-e^b) \lim_{n\rightarrow\infty}
\bigg(\frac{(a-b)}{n}\frac{1}{e^{\sfrac{(b-a)}{n}}-1}\bigg)
\end{align*}

für den Term $e^{\sfrac{(b-a)}{n}}$, wir benutzen Taylor-Entwicklung bei Punkt $\frac{b-a}{n}$ mit $n\rightarrow\infty$ (d.h. $\frac{b-a}{n}\rightarrow 0$)
\begin{align*}
e^{\sfrac{(b-a)}{n}}
\simeq
1+\frac{b-a}{n}+
\underset{ignore}{\underbrace{
\frac{1}{2!}
\bigg(\frac{b-a}{n}\bigg)^2+
\mathcal{O}\bigg(\bigg(\frac{b-a}{n}\bigg)^3\bigg)
}}
\end{align*}

\newpage

dann gilt
\begin{align*}
&(e^a-e^b) \lim_{n\rightarrow\infty}
\bigg(\frac{(a-b)}{n}\frac{1}{e^{\sfrac{(b-a)}{n}}-1}\bigg) \\
=&(e^a-e^b) &\lim_{n\rightarrow\infty}
\bigg(\frac{(a-b)}{n}\frac{1}{1+\frac{b-a}{n}-1}\bigg) \\
=&e^b-e^a
\end{align*}
\begin{align*}
\int_a^b e^x dx=e^b-e^a
\end{align*}

\end{enumerate}

\newpage

\paragraph{Aufgabe 20.6(H)}

\begin{enumerate}

\item[]

The function of the ellipse is given by
\begin{align*}
\frac{x^2}{a^2}+\frac{y^2}{b^2}\leq 1
\end{align*}
or
\begin{align*}
\frac{x^2}{a^2}+\frac{y^2}{b^2}=1
\end{align*}

therefore, we have
\begin{align*}
y=\frac{b}{a}\sqrt{a^2-x^2}
\end{align*}

to determine the area of the ellipse, we do integration over $y$
\begin{align*}
A=\int_{x_0}^{x_1} ydx
\end{align*}
with $x_0=0$, $x_1=a$. Note that the area $A$ is now only one-fourth of a ellipse, actual area is $A_e=4A$.

Then:
\begin{align*}
A=\int_{x_0}^{x_1} ydx=\int_{x_0}^{x_1} \frac{b}{a}\sqrt{a^2-x^2} dx=\frac{b}{a} \int_{x_0}^{x_1} \sqrt{a^2-x^2} dx
\end{align*}

now comes the hint given in the question: to find the anti-derivative of function $F'(x)=\sqrt{a^2-x^2}$, or to prove that for function $F(x)$
\begin{align*}
F(x)=\frac{1}{2}\bigg(x\sqrt{a^2-x^2}+a^2\arcsin\bigg(\frac{x}{a}\bigg)\bigg)
\end{align*}
its derivative is $F'(x)=\sqrt{a^2-x^2}$.

\begin{proof}

For function
\begin{align*}
F(x)
=&\frac{1}{2}\bigg(x\sqrt{a^2-x^2}+a^2\arcsin\bigg(\frac{x}{a}\bigg)\bigg) \\
=&\frac{1}{2}x\sqrt{a^2-x^2}+\frac{1}{2}a^2\arcsin\bigg(\frac{x}{a}\bigg)
\end{align*}

its derivative is
\begin{align*}
F'(x)
=&\frac{1}{2}\sqrt{a^2-x^2}-\frac{1}{2}\frac{x^2}{\sqrt{a^2-x^2}}+\frac{1}{2}a\frac{1}{\sqrt{1-\frac{x^2}{a^2}}} \\
=&\frac{1}{2}\bigg(
\frac{a^2-2x^2}{\sqrt{a^2-x^2}}+\frac{a^2}{\sqrt{a^2-x^2}}
\bigg) \\
=&\sqrt{a^2-x^2}
\end{align*}

\end{proof}

\newpage

Therefore, we have
\begin{align*}
\int F'(x) dx
&=
F(x)+C \\
\int \sqrt{a^2-x^2} dx
&=
\frac{1}{2}\bigg(x\sqrt{a^2-x^2}+a^2\arcsin\bigg(\frac{x}{a}\bigg)\bigg)
\end{align*}

now calculate the definite integral
\begin{align*}
&\int_{x_0}^{x_1} \sqrt{a^2-x^2} dx \\
=&
\frac{1}{2}\bigg(x\sqrt{a^2-x^2}+a^2\arcsin\bigg(\frac{x}{a}\bigg)\bigg)\bigg|_0^a \\
=&
\frac{1}{2}\pi a^2
\end{align*}

therefore
\begin{align*}
A=\frac{b}{a}\frac{1}{2}\pi a^2=\frac{1}{4}\pi ab
\end{align*}

the area of the ellipse is $A_e=4A=\pi ab$

\end{enumerate}
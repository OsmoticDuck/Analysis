\section{14. Übungsblatt}

\begin{align*}
\cos x&:=\frac{1}{2}(e^{ix}+e^{-ix})\\
\sin x&:=\frac{1}{2i}(e^{ix}-e^{-ix})\\
\sin(x+y)&=\sin x\cos y+\sin y\cos x
\end{align*}

\paragraph{Aufgabe 14.1}

\begin{enumerate}

\item[(a)]
\begin{equation*}
\mathrm{Re}z=\frac{z+\overline{z}}{2}
\end{equation*}
\begin{equation*}
\frac{1}{\mathrm{Re}z}=\frac{2}{z+\overline{z}}\neq\mathrm{Re}\frac{1}{z}
\end{equation*}
Gegenbsp. $z=1+i$

\item[(b)]

$|z|^2=z\overline{z}$

dann $z=\frac{\overline{z}}{|z|^2}$

$\mathrm{Im}\frac{1}{z}=\mathrm{Im}\frac{\overline{z}}{|z|^2}=\cdots$

\item[(c)]

$\mathrm{Re}(z+\frac{1}{z})=\mathrm{Re}z+\mathrm{Re}\frac{1}{z}=(1+\frac{1}{|z|^2})\mathrm{Re}z=0$

dann $\mathrm{Re}z=0$

\end{enumerate}

\paragraph{Aufgabe 14.2}

\begin{enumerate}

\item[(a)]

$z=x+iy$ betrachten

\item[(b)]

\begin{enumerate}

\item[(i)]

$x=\frac{\sqrt{3}}{2}$

\item[(ii)]

$\cos x-i\sin x=-e^{-ix}$ betrachten

\end{enumerate}

\end{enumerate}

\paragraph{Aufgabe 14.3}

\begin{enumerate}

\item[(a)]

$\cdots=(e^{ix})^n=e^{inx}=\square$

\item[(b)]

\begin{equation*}
(\cos x+i\sin x)^3=\cos3x+i\sin3x
\end{equation*}

Links:
\begin{align*}
\cos3x&=\cos^3x-3\sin^2x\cos x\\
\sin3x&=-\sin^3x+3\sin x\cos^2x
\end{align*}

\item[(c)]

$\tan x=\frac{\sin x}{\cos x}$ betrachten

\end{enumerate}

\newpage

\paragraph{Aufgabe 14.4}

\begin{enumerate}

\item[]

\begin{align*}
\sinh x&=\frac{e^x-e^{-x}}{2}\\
\cosh x&=\frac{e^x+e^{-x}}{2}
\end{align*}

\begin{align*}
\cos(x+iy)
&=\frac{1}{2}(e^{i(x+iy)}+e^{-i(x+iy)})\\
&=\frac{1}{2}(e^{ix-y}+e^{-ix+y})\\
&=\frac{1}{2}(e^{-y}(\cos x+i\sin x)+e^y(\cos x-i\sin x))\\
&=\frac{1}{2}(\cos x(e^{-y}+e^y)+i\sin x(e^{-y}-e^y))\\
&=\cos x\frac{e^y+e^{-y}}{2}-i\sin x\frac{e^y-e^{-y}}{2}\\
&=\square
\end{align*}

\end{enumerate}
\section{4. Übungsblatt:}

\subsection{Aufgabe 4.1}

\paragraph{(a)}
$\begin{pmatrix}
2\\
5
\end{pmatrix}$
\begin{equation*}
\begin{pmatrix}
2\\
5
\end{pmatrix}
=\frac{2-4}{5}
\begin{pmatrix}
2\\
4
\end{pmatrix}
=-\frac{2}{5}\frac{2-3}{4}
\begin{pmatrix}
2\\
3
\end{pmatrix}
=\frac{2}{5}\frac{1}{4}\frac{2-2}{3}
\begin{pmatrix}
2\\
2
\end{pmatrix}
=0
\end{equation*}

\paragraph{(b)}
$\begin{pmatrix}
-1\\
k
\end{pmatrix}$

\begin{equation*}
\begin{pmatrix}
-1\\
0
\end{pmatrix}=1,
\begin{pmatrix}
-1\\
1
\end{pmatrix}=-1,
\begin{pmatrix}
-1\\
2
\end{pmatrix}=1,
\end{equation*}
Beh. $\begin{pmatrix}
-1\\
k
\end{pmatrix}=(-1)^k$, mit Induktion.
\begin{proof}
$ $\newline

IA:..\\

IV:..\\

IS:$\begin{pmatrix}
-1\\
k+1
\end{pmatrix}$
\begin{equation*}
\begin{pmatrix}
-1\\
k+1
\end{pmatrix}=\frac{-1-k}{k+1}
\begin{pmatrix}
-1\\
k
\end{pmatrix}=(-1)^{k+1}
\end{equation*}
\end{proof}

\paragraph{(c)}
$\begin{pmatrix}
-\frac{1}{2}\\
3
\end{pmatrix}$
\begin{equation*}
\begin{pmatrix}
-\frac{1}{2}\\
3
\end{pmatrix}
=\frac{-\frac{1}{2}-2}{3}
\begin{pmatrix}
-\frac{1}{2}\\
2
\end{pmatrix}
=-\frac{5}{6}\frac{-0.5-1}{2}
\begin{pmatrix}
-0.5\\
1
\end{pmatrix}
=\frac{5}{6}\frac{3}{4}\frac{-0.5-0}{1}
\begin{pmatrix}
-0.5\\
0
\end{pmatrix}
=-\frac{5}{16}
\end{equation*}

\newpage

\subsection{Aufgabe 4.2}

\paragraph{(a)}
z.z. $\begin{pmatrix}
x\\
k
\end{pmatrix}
=
\begin{pmatrix}
x-1\\
k-1
\end{pmatrix}\frac{x}{k}$
\begin{proof}
$ $\newline

IA: $k=1$, LS$=
\begin{pmatrix}
x\\
1
\end{pmatrix}
=\frac{x-0}{1}
\begin{pmatrix}
x\\
0
\end{pmatrix}
=\frac{x}{1}
\begin{pmatrix}
x-1\\
0
\end{pmatrix}$

IV: für $k$ gilt
\begin{equation*}
\begin{pmatrix}
x\\
k
\end{pmatrix}
=\frac{x}{k}
\begin{pmatrix}
x-1\\
k-1
\end{pmatrix}
\end{equation*}

Ziel:
\begin{equation*}
\begin{pmatrix}
x\\
k+1
\end{pmatrix}
=\frac{x}{k+1}
\begin{pmatrix}
x-1\\
k
\end{pmatrix}
\end{equation*}

IS:
\begin{align*}
\begin{pmatrix}
x\\
k+1
\end{pmatrix}
=\frac{x}{k+1}
\begin{pmatrix}
x\\
k
\end{pmatrix}
&\overset{\mathbf{IV}}{=}
\frac{x-k}{k+1}\frac{x}{k}
\begin{pmatrix}
x-1\\
k-1
\end{pmatrix}\\
&=\frac{x}{k+1}\frac{x-k}{k}\frac{k}{x-k}\frac{x-k}{k}
\begin{pmatrix}
x-1\\
k-1
\end{pmatrix}\\
&=\frac{x}{k+1}
\begin{pmatrix}
x-1\\
k
\end{pmatrix}
\end{align*}
\end{proof}

\paragraph{(b)}
z.z. $
\begin{pmatrix}
x+1\\
k
\end{pmatrix}=
\begin{pmatrix}
x\\
k-1
\end{pmatrix}+
\begin{pmatrix}
x\\
k
\end{pmatrix}
$
\begin{proof}
\begin{equation*}
\begin{pmatrix}
x+1\\
k
\end{pmatrix}
=\frac{x+1}{k}
\begin{pmatrix}
x+1-1\\
k-1
\end{pmatrix}
=\frac{x+1+k-k}{k}
\begin{pmatrix}
x\\
k-1
\end{pmatrix}
=
\underset{
\begin{pmatrix}
x\\
k
\end{pmatrix}
}{\underbrace{
\frac{\overset{x-(k-1)}{\overbrace{x+1-k}}}{k}
\begin{pmatrix}
x\\
k-1
\end{pmatrix}
}}
+
\begin{pmatrix}
x\\
k-1
\end{pmatrix}
\end{equation*}
\end{proof}

\newpage

\subsection{Aufgabe 4.3}

\begin{equation*}
x^0:=1,\ x^{n+1}=x^{n}\cdot x
\end{equation*}

\paragraph{(a)}
\begin{proof}
z.z. $x^mx^n=x^{m+n}$\\

Sei $\forall n\in\mathbb{N}_0$ fest $\rightarrow$ Induktion.\\

IA: $n=0$: $x^mx^n=x^m\cdot 1=x^m=x^{m+n}$\\

IV: $x^mx^n=x^{m+n}$\\

Ziel: $x^mx^{n+1}=x^{m+n+1}$\\

IS: LS$=x^m(x^nx)\overset{Ass.Mult}{=}(x^mx^n)x=x^{m+n}x=x^{(m+n)+1}=x^{m+(n+1)}$
\end{proof}

\paragraph{(b)}

z.z. $(x^m)^n=x^{mn}$\\

\begin{proof}
$ $\newline

Sei $\forall m\in\mathbb{N}_0$ fest\\

IA: $n=0$,: $(x^m)^0=1=x^0=x^{m\cdot 0}$\\

IV: $(x^m)^n=x^{mn}$\\

Ziel: $(x^m)^{n+1}=x^{m(n+1)}$\\

IS: $(x^m)^{n+1}=(x^m)^n\cdot x^m\overset{\mathbf{IV}}{=}x^{mn}x^n\overset{(a)}{=}x^{mn+n}$\\
\end{proof}

\paragraph{(c)}
\begin{proof}
$ $\newline

IA: $n=0$ $\rightarrow$ $(x\cdot y)^n=1=1\cdot 1=x^0y^0$\\

IV: ......\\

IS: $(xy)^{n+1}=(xy)^n(xy)\overset{\mathbf{IV}}{=}(x^ny^n)(xy)\overset{Field-Axiom}{=}(x^nx)(y^ny)=x^{n+1}y^{n+1}$
\end{proof}


\newpage

\subsection{Aufgabe 4.4}
See definition of Group, Monoid etc.

\newpage

\subsection{Aufgabe 4.5}

Inverse: Sei $(G,\cdot)$ Gruppe\\
Man sagte dass $f\in G$ ist Inverse von $g\in G$\\
$\overset{Def}{\Leftrightarrow}$ $f\cdot g=g\cdot f=e$, Notation: $f=:g^{-1}$

\paragraph{(a)}
\begin{proof}
$ $\newline

z.z. $(a^{-1})^{-1}=a$ falls $a\neq 0$\\

$\Leftrightarrow$ $a$ ist Inv. von $a^{-1}$ $\Leftrightarrow$ $a\cdot a^{-1}=a^{-1}\cdot a=e$\\

Das gilt da $a$ Inv. von $a^{-1}$
\end{proof}

\paragraph{(b)}
\begin{proof}
$ $\newline

z.z. $(ab)^{-1}=a^{-1}b^{-1}$\\

$(ab)(a^{-1}b^{-1})\overset{F.Axiom}{=}(a\cdot a^{-1})(b\cdot b^{-1})=1$\\

$(a^{-1}b^{-1})(ab)\overset{F.Axiom}{=}(a^{-1}\cdot a)(b^{-1}\cdot b)=1$\\
\end{proof}

\newpage

\subsection{Aufgabe 4.6(H)}

\paragraph{(a)}
\begin{proof}
$ $\newline

z.z. $\forall n\in\mathbb{N}_{\geq1}$, es gilt
\begin{equation*}
\sum_{k=0}^{n}\frac{(-1)^k}{k+1}
\begin{pmatrix}
n \\
k
\end{pmatrix}
=
\frac{1}{n+1}
\end{equation*}

Wir haben
\begin{align*}
\begin{pmatrix}
n \\
k
\end{pmatrix}
&=
\frac{n}{k}
\begin{pmatrix}
n-1 \\
k-1
\end{pmatrix}\\
\begin{pmatrix}
n+1 \\
k+1
\end{pmatrix}
&=
\frac{n+1}{k+1}
\begin{pmatrix}
n \\
k
\end{pmatrix}\\
\frac{1}{n+1}
\begin{pmatrix}
n+1 \\
k+1
\end{pmatrix}
&=
\frac{1}{k+1}
\begin{pmatrix}
n \\
k
\end{pmatrix}
\end{align*}

betrachten
\begin{align*}
\sum_{k=0}^{n}\frac{(-1)^k}{n+1}
\begin{pmatrix}
n+1 \\
k+1
\end{pmatrix}
&=
\frac{1}{n+1}\\
\frac{1}{n+1}
\sum_{k=0}^{n}(-1)^k
\begin{pmatrix}
n+1 \\
k+1
\end{pmatrix}
&=
\frac{1}{n+1}\\
\end{align*}

dann, z.z. $\forall n\in\mathbb{N}_{\geq1}$
\begin{equation*}
\sum_{k=0}^{n}(-1)^k
\begin{pmatrix}
n+1 \\
k+1
\end{pmatrix}
=
1
\end{equation*}

IA: Sei $n=1$, es gilt
\begin{equation*}
\sum_{k=0}^{1}(-1)^k
\begin{pmatrix}
1+1 \\
k+1
\end{pmatrix}
=
1
\begin{pmatrix}
2 \\
1
\end{pmatrix}+
(-1)
\begin{pmatrix}
2 \\
2
\end{pmatrix}
=
2-1
=
1
\end{equation*}

IV: $\forall n\in\mathbb{N}_{\geq1}$, es gilt
\begin{equation*}
\sum_{k=0}^{n}(-1)^k
\begin{pmatrix}
n+1 \\
k+1
\end{pmatrix}
=
1
\end{equation*}

IS: Sei $\tilde{n}=n+1$, wir betrachten
\begin{align*}
\sum_{k=0}^{\tilde{n}}(-1)^k
\begin{pmatrix}
\tilde{n}+1 \\
k+1
\end{pmatrix}
&=
\sum_{k=0}^{n+1}(-1)^k
\begin{pmatrix}
n+2 \\
k+1
\end{pmatrix}\\
&=
\sum_{k=0}^{n+1}(-1)^k
\begin{pmatrix}
n+1 \\
k+1
\end{pmatrix}+
\sum_{k=0}^{n+1}(-1)^k
\begin{pmatrix}
n+1 \\
k
\end{pmatrix}\\
&=
\sum_{k=0}^{n}(-1)^k
\begin{pmatrix}
n+1 \\
k+1
\end{pmatrix}+
(-1)^{n+1}
\begin{pmatrix}
n+1 \\
n+2
\end{pmatrix}+
\sum_{k=0}^{n}(-1)^k
\begin{pmatrix}
n+1 \\
k
\end{pmatrix}+
(-1)^{n+1}
\begin{pmatrix}
n+1 \\
n+1
\end{pmatrix}\\
&\overset{\mathbf{IV}}{=}
1+0+
\sum_{k=0}^{n}(-1)^k
\begin{pmatrix}
n+1 \\
k
\end{pmatrix}+
(-1)^{n+1}\\
&=1+0+0=1
\end{align*}
\end{proof}

\newpage

\paragraph{(b)}
$ $\newline

Vermutung: $\forall n\in\mathbb{N}_{\geq1}$ und $m\in\mathbb{N}_{\geq 1}$
\begin{equation*}
\sum_{k=1}^{n}\prod_{i=0}^{m-1}(k+i)=\frac{1}{m+1}\prod_{i=0}^{m}(n+i)
\end{equation*}
\begin{proof}
$ $\newline

IA: Sei $n=1$, es gilt
\begin{equation*}
\sum_{k=1}^{1}\prod_{i=0}^{m-1}(k+i)=\prod_{i=0}^{m-1}(1+i)=\frac{1}{1+m}\prod_{i=0}^{m}(1+i)
\end{equation*}

IV: $\forall n\in\mathbb{N}_{\geq1}$ und $m\in\mathbb{N}_{\geq 1}$
\begin{equation*}
\sum_{k=1}^{n}\prod_{i=0}^{m-1}(k+i)=\frac{1}{m+1}\prod_{i=0}^{m}(n+i)
\end{equation*}

IS: Sei $\tilde{n}=n+1$, wir betrachten
\begin{align*}
\sum_{k=1}^{\tilde{n}}\prod_{i=0}^{m-1}(k+i)
&=
\sum_{k=1}^{n+1}\prod_{i=0}^{m-1}(k+i)\\
&=
\sum_{k=1}^{n}\prod_{i=0}^{m-1}(k+i)+
\prod_{i=0}^{m-1}(n+1+i)\\
&\overset{\mathbf{IV}}{=}
\frac{1}{m+1}\prod_{i=0}^{m}(n+i)+
\prod_{i=0}^{m-1}(n+1+i)\\
&=
\frac{1}{m+1}\prod_{i=0}^{m}(n+i)+
\frac{1}{n}\prod_{i=0}^{m}(n+i)\\
&=
\Big(\frac{1}{m+1}+\frac{1}{n}\Big)\prod_{i=0}^{m}(n+i)\\
&=
\frac{n+m+1}{n(m+1)}\prod_{i=0}^{m}(n+i)\\
&=
\frac{1}{m+1}\frac{n+m+1}{n}\prod_{i=0}^{m}(n+i)\\
&=
\frac{1}{m+1}\prod_{i=0}^{m}(n+1+i)
\end{align*}
\end{proof}

\newpage

\subsection{Aufgabe 4.7(H)}
\begin{proof}
$ $\newline

IA: Sei $n=0$, es gilt
\begin{equation*}
\sum_{k=0}^{0}(-1)^k\cdot
\begin{pmatrix}
x \\
k
\end{pmatrix}
=
(-1)^0\cdot
\begin{pmatrix}
x \\
0
\end{pmatrix}
=
1
=
(-1)^0\cdot
\begin{pmatrix}
x-1 \\
0
\end{pmatrix}
\end{equation*}

IV: $x\in\mathbb{R}$, $n\in\mathbb{N}$
\begin{equation*}
\sum_{k=0}^{n}(-1)^k\cdot
\begin{pmatrix}
x \\
k
\end{pmatrix}
=
(-1)^n\cdot
\begin{pmatrix}
x-1 \\
n
\end{pmatrix}
\end{equation*}

IS: Sei $\tilde{n}=n+1$, wir betrachten
\begin{align*}
\sum_{k=0}^{\tilde{n}}(-1)^k\cdot
\begin{pmatrix}
x \\
k
\end{pmatrix}
&=
\sum_{k=0}^{n+1}(-1)^k\cdot
\begin{pmatrix}
x \\
k
\end{pmatrix}\\
&=
\sum_{k=0}^{n}(-1)^k\cdot
\begin{pmatrix}
x \\
k
\end{pmatrix}+
(-1)^{n+1}\cdot
\begin{pmatrix}
x \\
n+1
\end{pmatrix}\\
&\overset{\mathbf{IV}}{=}
(-1)^n\cdot
\begin{pmatrix}
x-1 \\
n
\end{pmatrix}+
(-1)^{n+1}\cdot
\begin{pmatrix}
x \\
n+1
\end{pmatrix}\\
&=
-(-1)^{n+1}\cdot
\begin{pmatrix}
x-1 \\
n
\end{pmatrix}+
(-1)^{n+1}\cdot
\begin{pmatrix}
x \\
n+1
\end{pmatrix}\\
&=
(-1)^{n+1}\cdot\Bigg(
\begin{pmatrix}
x \\
n+1
\end{pmatrix}-
\begin{pmatrix}
x-1 \\
n
\end{pmatrix}
\Bigg)\\
&=
(-1)^{n+1}\cdot\Bigg(
\begin{pmatrix}
x-1 \\
n+1
\end{pmatrix}+
\begin{pmatrix}
x-1 \\
n
\end{pmatrix}-
\begin{pmatrix}
x-1 \\
n
\end{pmatrix}
\Bigg)\\
&=
(-1)^{n+1}\cdot
\begin{pmatrix}
x-1 \\
n+1
\end{pmatrix}
\end{align*}
\end{proof}

\newpage

\subsection{Tutorium}

\textsc{Gauss Klammern} oder \textsc{Floor and Ceiling Function}\\

$\forall x,y\in\mathbb{R}$, $x-y\in\mathbb{Z}$\\

$\Rightarrow$ $x-\lfloor x\rfloor=y-\lfloor y\rfloor$\\

"$\Leftarrow$" $x-y=\lfloor x\rfloor-\lfloor y\rfloor$\\

$\Rightarrow$ $x-y\in\mathbb{Z}$\\

"$\Rightarrow$" $x-y\in\mathbb{Z}$\\

$\Leftrightarrow$ $\exists k\in\mathbb{Z}$: $x-y=k$\\

$\Rightarrow$ $x-\lfloor x\rfloor-(y-\lfloor y\rfloor)=$\\
